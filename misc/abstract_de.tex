\thispagestyle{empty}
\vspace*{0.2cm}

\begin{center}
    \textbf{Zusammenfassung}
\end{center}

\vspace*{0.2cm}

% \noindent

\textit{Hysterese} ist definiert als ein \textit{raten unabh{\"a}ngiger} Prozess mit lokalem \textit{Speicher}. Eine breite Klasse von hysteretischen Systemen wird durch einen Elementarblock \textit{Preisach-Operator} gebildet. In dieser Arbeit approximieren wir diese Art von hysteretischem Prozess durch neuronale Netze.
RNNs k{\"o}nnen Prozesse mit Speicher approximieren, aber Standardarchitekturen wie LSTM k{\"o}nnen die hysteretische Beziehung nicht aus Ein- und Ausg{\"a}ngen lernen.
Daher entwickeln wir eine neue Netzwerkarchitektur, sogenannte hysteretische neuronale Netze (HNN), die ein {\"U}berlagerung von \textit{nichtlinearen Spielen} ist. Wir vergleichen den RMSE zwischen HNN- und LSTM-Netzwerken und es zeigt sich, dass HNN signifikant besser ist als LSTM-Netzwerke.
Dar{\"u}ber hinaus verallgemeinern wir ein Finanzmarktmodell, das auf \citep{dima2014} basiert.
Es zeigt auch, dass HNN mit weniger Parametern eine bessere Leistung erzielen kann als LSTM-Netzwerke.