% http://pgfplots.sourceforge.net/gallery.html
% https://tex.stackexchange.com/questions/120441/plot-the-probability-density-function-of-the-gamma-distribution
\documentclass[border=5pt]{standalone}
\usepackage{pgfplots}
\usepackage{pgfplotstable}
    % \pgfplotsset{
    %     % use this `compat' level or higher to position the bars in one group
    %     % next to each other
    %     compat=1.16,
    % }
    % % load the data table ...
    % \pgfplotstableread[col sep=comma]{virtual-agentn-distribution.csv}{\loadedtable}
        % and store the number of columns in `\NoOfCols'
        % (minus 1 because counting in `\foreach' starts with zero
        %\pgfplotstablegetcolsof{\loadedtable}
        % \pgfmathtruncatemacro{\NoOfCols}{\pgfplotsretval-1}
        
        
        
\begin{filecontents}{data.csv}
2.0,1
2.5,2
3.0,3
3.5,6
4.0,10
4.5,13
5.0,17
5.5,20
6.0,22
6.5,23
7.0,24
7.5,23
8.0,22
8.5,21
9.0,19
9.5,17
10.0,14
10.5,12
11.0,10
11.5,9
12.0,7
12.5,6
13.0,5
13.5,4
14.0,3
14.5,2
15.0,2
15.5,1
16.0,1
16.5,1
17.0,1
\end{filecontents}    
% \pgfplotstableread[col sep=comma]{virtual-agentn-distribution.csv}{\loadedtable}
\pgfplotstableread[col sep=comma]{data.csv}{\loadedtable}

\usepackage{pst-func}
\usepackage{auto-pst-pdf}

\begin{document}
\begin{tikzpicture}[ declare function={gamma(\z)=
    (2.506628274631*sqrt(1/\z) + 0.20888568*(1/\z)^(1.5) + 0.00870357*(1/\z)^(2.5) - (174.2106599*(1/\z)^(3.5))/25920 - (715.6423511*(1/\z)^(4.5))/1244160)*exp((-ln(1/\z)-1)*\z);},
    declare function={gammapdf(\x,\k,\theta) = \x^(\k-1)*exp(-\x/\theta) / (\theta^\k*gamma(\k+1)/\k);}
]
    \begin{axis}[
        % adjust the `width' a bit by keeping the default `height'
        width=2*\axisdefaultwidth,
        height=\axisdefaultheight,
        % set appropriate `ymax' value so the `nodes near coords' fit to the plot
        % (adjusting the `ymin' value is just to make it look a little bit better)
        ymin=0,
        ymax=25,
        % there should be no gap between the bars in one group
        ybar=0pt,
        xlabel=blance,
        ylabel={\# Real agents N},
        % use data from the table for the xticklabels
        xtick=data,
        xticklabel style={font=\tiny, xshift=0.3ex},
        % xticklabels from table={\loadedtable}{COUNT},
        % to start the bars from the bottom edge of the plot
        % (otherwise they would start from 10^0
        %  borrowed from <http://tex.stackexchange.com/a/86688/95441)
        % log origin=infty,
        % adjust the size of the bars so they don't overlap
        % (you can play with the numerator to change the gap between the groups)
        bar width=0.5,
        % enlarge the x limits so all of the bars are shown
        % (play with the value to adjust the gap on the sides of the plot)
        % enlarge x limits={abs=0.6},
        % and position the legend outside of the plot to not overlap with the data
        % legend pos=outer north east,
        % add `nodes near coords'
        nodes near coords={
            % because internally PGFPlots works with floating point numbers, we
            % change them to fixed point numbers
            % \pgfkeys{
            %     /pgf/fpu=true,
            %     /pgf/fpu/output format=fixed,
            % }
            %
            % % check if numbers are greater than 1000 and if so, divide them by
            % % 1000 to convert them from ms to s scale
            % \pgfmathparse{
            %     ifthenelse(
            %         \pgfplotspointmeta < 1000,
            %         \pgfplotspointmeta,
            %         \pgfplotspointmeta/1000
            %     )
            % }%
            % % to now decide which of the two cases we have, we compare the
            % % point meta value, but because `\ifnum' compares integers, we first
            % % have to convert the fixed number to an integer
            %     \pgfmathtruncatemacro{\Y}{\pgfplotspointmeta}%
            % \ifnum\Y<1000
            %     \pgfmathprintnumber{\pgfmathresult}
            % \else
            %     \pgfmathprintnumber{\pgfmathresult}
            % \fi
        },
        % set the style of the `nodes near coords'
        % nodes near coords style={
        %     font=\tiny,
        %     rotate=90,
        %     anchor=west,
        % },
        % as basis for the `nodes near coords' use the raw y values
        % point meta=rawy,
    ]
        {
             \addplot table[x index={0}, y index={1}]{\loadedtable};
             \addplot [smooth, domain=0:20, red] {160*gammapdf(x,8,1.0)};
             \addlegendentry{Real agents N distribution}
             \addlegendentry{$\frac{x^{k-1} e^{-x/\theta} }{\theta^{k}\Gamma(k)}, k=8, \theta=1$}
        }
    \end{axis}
\end{tikzpicture}
\end{document}