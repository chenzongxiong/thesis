\documentclass{article}

\usepackage{tikz}
\begin{document}
\pagestyle{empty}

\def\layersep{2.5cm}

\begin{tikzpicture}[shorten >=1pt,->,draw=black!50, node distance=\layersep]
    \tikzstyle{every pin edge}=[<-,shorten <=1pt]
    \tikzstyle{neuron}=[circle,fill=black!25,minimum size=17pt,inner sep=0pt]
    \tikzstyle{input neuron}=[neuron, fill=green!50];
    \tikzstyle{output neuron}=[neuron, fill=red!50];
    \tikzstyle{hidden neuron}=[neuron, fill=blue!50];
    \tikzstyle{annot} = [text width=4em, text centered]

    % Draw the input layer nodes
    \foreach \name / \y in {1/1,2/2,3/3,6/n}
    \node[input neuron, pin=left:Input \#\y] (I-\name) at (0,-\y) {$x_\y$};

    % Draw the hidden layer nodes
    \foreach \name / \y in {1/1,2/2,3/3,6/n}
    % \path[yshift=0.5cm]
    node[hidden neuron] (H-\name) at (\layersep,-\y cm) {$p_\y$};

    % % Draw the output layer node
    % \node[output neuron,pin={[pin edge={->}]right:Output}, right of=H-3] (O) {};

    % % Connect every node in the input layer with every node in the
    % % hidden layer.
    % \foreach \source in {1,...,4}
    %     \foreach \dest in {1,...,5}
    %         \path (I-\source) edge (H-\dest);

    % % Connect every node in the hidden layer with the output layer
    % \foreach \source in {1,...,5}
    %     \path (H-\source) edge (O);

    % % Annotate the layers
    % \node[annot,above of=H-1, node distance=1cm] (hl) {Hidden layer};
    % \node[annot,left of=hl] {Input layer};
    % \node[annot,right of=hl] {Output layer};
\end{tikzpicture}
% End of code
\end{document}

% \documentclass{article}

% \usepackage{tikz}
% \usetikzlibrary{calc}

% \begin{document}
% \pagestyle{empty}

% \def\layersep{2.5cm}

% \tikzset{neuron/.style={circle,thick,fill=black!25,minimum size=17pt,inner sep=0pt},
%   input neuron/.style={neuron, draw,thick, fill=gray!30},
%   hidden neuron/.style={neuron,fill=white,draw},
%   hoz/.style={rotate=-90}}   %<--- for labels

% \begin{tikzpicture}[-,draw=black, node distance=\layersep,transform shape,rotate=90]  %<-- rotate the NN

%   % Draw the input layer nodes
%   \foreach \name / \y in {1/1,2/2,3/3,6/n}
%   \node[input neuron, hoz] (I-\name) at (0,-\name) {\color{red}$x_\y$};

%   \node[hoz] (I-4) at (0,-4) {$\dots$};

%   % \foreach \name / \y in {1/1,2/2,3/3,5/m}
%   % \path[hoz] (I-\name) node[below=0.5cm](0,-\name) {$v_\y$};

%   % Draw the hidden layer nodes
%   \foreach \name / \y in {1/1,2/2,3/3,6/n}
%   \path[yshift=0.5cm] node [hidden neuron, hoz] (H-\name) at (\layersep,-\name cm) {\color{red}$p_\y$};

%   \path[yshift=0.5cm]
%   node[hoz] () at (\layersep,-4 cm) {$\dots$};
%   \path[yshift=0.5cm]
%   node[hoz] () at (\layersep,-5 cm) {$\dots$};

%   % \foreach \name / \y in {1/1,2/2,3/3,6/n}
%   % \path[hoz] (H-\name) node[above=0.5cm]  {$h_\y$};

%   % \path node[hoz,right] at ($(I-5)!0.5!(H-6)$) {\color{red}$w_{nm}$};

%   % Connect every node in the input layer with every node in the  hidden layer.
%   \foreach \source in {1,2,3,6}
%   \foreach \dest in {1,2,3,6}
%   \path (I-\source.north) edge (H-\dest.south);
% \end{tikzpicture}
% % End of code
% \end{document}
