\section{Micro experiments}
Before we compare the performance of HNN with other recurrent networks, we confirm that the effects of HNN's initial states can be ignored if we have enough large data sets. 
% Also, we will show the fact that $G^{-1}$ can be approximated by $F$ within tolerant error.

\subsection{Initial states}\label{sec:chapter5:initial-states}
We use data sets $D$ and $P$ and set different initial states during the predicting phase.
\mytableref{tbl:chapter5:initial-states} presents the influence of different initial states. We mark a result in bold if the initial state is correct throughout prediction. We see the incorrect initial states do make RMSE worse. But these impacts can be ignored after 150 timesteps in set $P$. For set $D$, it takes nearly 300 timesteps to synchronize the internal states of each \textit{play}. Comparing the results of sets $D$ and $P$, we see that the different data sets would take different timesteps to synchronize the internal states. Even though the incorrect states $1$ and $100$ in set $D$, they can produce the same RMSE. Mainly, it's because of the next movement in their dynamics are the same even though they are at different states (see \myfigref{fig:chapter1:play}).
 
From \myfigref{fig:chapter5:initial-states}, we see that the predictive curves with different initial states diverge at the very beginning steps of predictions. However, they are overlapping exactly within finite timesteps. 

In the following analysis, we always set the correct initial states in predicting phase.

\begin{table}[htb!]
\centering
\begin{adjustbox}{angle=0}
\begin{tabular}{||c|c|c|c|c|c||}
\hline 
    Data sets & Initial states & RMSE & RMSE(50) & RMSE(150) & RMSE(300)
    % & RMSE(200) 
    \\
\hline \hline
\multirow{4}{4em}{$D$} & 1 & 2.78 $\pm$ 0.32 & 2.97 $\pm$ 0.34 & 3.52 $\pm$ 0.40 & 5.56 $\pm$
0.64 \\
                           & 100 & 2.78 $\pm$ 0.32 & 2.97 $\pm$ 0.34 & 3.52 $\pm$ 0.40  &5.56 $\pm$
0.64  \\
                           & -1 &  5.49 $\pm$ 1.14 & 5.08 $\pm$ 1.03  & 4.72 $\pm$ 0.70 &5.56 $\pm$
0.64  % & 4.69 $\pm$ 0.43
                           \\
                           & -100 & 5.49 $\pm$ 1.14 & 5.08 $\pm$ 1.03 & 4.72 $\pm$ 0.70 & 5.56 $\pm$
0.64 % & 4.69 $\pm$ 0.43 
                           \\
                           & \textbf{correct} & \textbf{2.78 $\pm$ 0.32} & \textbf{2.97 $\pm$ 0.34} & \textbf{3.52 $\pm$ 0.40}  &  \textbf{5.56 $\pm$
0.64} 
                           % & \textbf{3.93 $\pm$ 0.45} 
                           \\
\hline

\multirow{4}{4em}{$P$} & 1 & 4.97 $\pm$ 0.04 &  2.52 $\pm$ 0.03 & 0.15 $\pm$ 0.03 & - \\
                           & 100 & 4.97 $\pm$ 0.04 & 2.52 $\pm$ 0.03 & 0.15 $\pm$ 0.03 & - \\
                           & -1 & 0.69 $\pm$ 0.04 & 0.51 $\pm$ 0.03 & 0.15 $\pm$ 0.03  & - \\
                           & -100 & 2.90 $\pm$ 0.03 & 1.20 $\pm$ 0.01 & 0.15 $\pm$ 0.03  & - \\
                           & \textbf{correct} & \textbf{0.15 $\pm$ 0.04} & \textbf{0.15 $\pm$ 0.03} & \textbf{0.15 $\pm$ 0.03} & - \\
\hline
\hline
\end{tabular}
\end{adjustbox}
\caption{Influence of initial state for predictions}
\label{tbl:chapter5:initial-states}
\end{table}


\begin{figure}[h!]
    \centering
    \subfloat[]{
        \includegraphics[height=4cm,width=\textwidth]{thesis/img/debug-initial-state-dima-seq.pdf}
    }
    \hfill
    \subfloat[]{
    \includegraphics[height=4cm,width=\textwidth]{thesis/img/debug-initial-state-pavel-seq.pdf}
    }
    \caption{Different initial states for set $D$ and $P$}
    \label{fig:chapter5:initial-states}
\end{figure}
\FloatBarrier