\section{Predicting price}
\textbf{Methods.} To distinguish difference predicting performance among different methods, we consider the following three approaches. The first approach is to predict the next price $\hat{p}_i$ as the previous price $p_{i-1}$, and we call it \textit{native method}. The second method is using LSTM networks described in \citep{kagglelstm}. The last one is using the methods described in \mysectionref{sec:chapter4:predicting_price}.  
 
\textbf{Measure.} To show the fine performance of the predictions, we divide \myupdate{price change} ${p_n-p_{n-1}}$ and \myupdate{noise jumps $b_n - b_{n-1}$} into $R \times R$ grids 
\mydelete{and generate the difference $|\hat{p}_i - p_i|$ at each}
slot $\text{SLOT}_j$ \myupdate{where}, 


\begin{equation}
\begin{aligned}
\text{SLOT}_j = \big\{ (\Delta p, \Delta b) : \Delta p_{j-1} < \Delta p \le \Delta p_j, \Delta b_{j-1} < \Delta b \le \Delta b_j \big\}
\end{aligned}
\end{equation}
and $\hat{p}_i$ is the predicted price, $p_i$ is the ground-truth price, $\Delta{p}_j = j \frac{\max\{p_1, p_2, \ldots\} - \min\{p_1, p_2, \ldots\}}{R}$, $\Delta{b}_j = j \frac{\max\{b_1, b_2, \ldots\} - \min\{b_1, b_2, \ldots\}}{R}$ and $j=0,1,\ldots$

Then the modified RMSE at 
$\text{SLOT}_j$ is given by,
\begin{equation}
\begin{aligned}
& & & \text{RMSE}(\text{SLOT}_j) = \sqrt{\frac{1}{C} \sum_{i} (\hat{p}_i - p_i)^2 }  \\
& \text{s.t.} & & (|p_i - p_{i-1}|, |b_i - b_{i-1}|) \in \text{SLOT}_j \\
& & & C = \text{card}(\text{SLOT}_j)
\end{aligned}
\end{equation}
\myupdate{With revised \text{RMSE}, we can easily explore how the noise jumps affects the predicted price accuracy.}

\textbf{Results.} \myfigref{fig:chapter5:predicting-price} shows the different RMSE for three methods mentioned above. 
Comparing \myfigref{fig:chapter5:predicting-price-baseline} and \myfigref{fig:chapter5:predicting-price-hnn}, we see that the RMSE are close to each other when $\Delta p < 0.059$. \myupdate{But} HNN performs better than the native methods when $\Delta p > 0.059$ (\myupdate{The RMSE of HNN and native approach are 0.082 and 0.097 respectively}). It indicates HNN can predict the dramatic fluctuation of the price whereas the native approach cannot.
We observe that the results generated by HNN have smaller RMSE among most slots in \myfigref{fig:chapter5:predicting-price-lstm,fig:chapter5:predicting-price-hnn}. As for $\Delta p >= 0.059$, LSTM exceeds HNN if we only consider RMSE. However, when we inspect the predicted price obtained by HNN and LSTM, we find that HNN gives us an insight for the avalanche of the price (see \myfigref{fig:chapter5:predicting-price,fig:chapter5:price-predictions,fig:chapter5:predicting-price-2}). Briefly explanation, we see there might be an avalanche from \myfigref{fig:chapter5:predicting-price-156} if external agents sell stocks to the market. \myupdate{And in the consecutive time steps, we do observe circumstance during the simulation \myfigref{fig:chapter5:predicting-price-157}.} However, LSTM is unaware to this situations. Instead, HNN captures this signal and predicts that the price would rise in the future. And \myfigref{fig:chapter5:predicting-price-158-price} proves the predictions. However, RMSE cannot reflect this situation well and that is why the RMSE of HNN is higher than that of LSTM when $\Delta b_n$ is larger ($\Delta b_n \ge  26.2$ in the \myfigref{fig:chapter5:predicting-price}).

\begin{figure}[htb!]
    \centering
    \subfloat[Native]{
    \includegraphics[height=4cm,width=\textwidth/3]{thesis/img/predictions/baseline-rmse.png}
        \label{fig:chapter5:predicting-price-baseline}

    }
     \subfloat[LSTM]{
    \includegraphics[height=4cm,width=\textwidth/3]{thesis/img/predictions/lstm-rmse.png}
        \label{fig:chapter5:predicting-price-lstm}
    
    }
     \subfloat[HNN]{
    \includegraphics[height=4cm,width=\textwidth/3]{thesis/img/predictions/hnn-rmse.png}
            \label{fig:chapter5:predicting-price-hnn}

    }
    \caption[The RMSE for different methods.]{The RMSE for different methods. The x-axis is $\Delta p$, the y-axis is $\Delta b$, and the z-axis is RMSE.}
    \label{fig:chapter5:predicting-price}
\end{figure}


\begin{figure}[ht!]
    \centering
    \subfloat[]{
    \includegraphics[height=4cm,width=\textwidth/2]{thesis/img/price-predictions/156.pdf}
        \label{fig:chapter5:predicting-price-156}

    }
     \subfloat[]{
    \includegraphics[height=4cm,width=\textwidth/2]{thesis/img/price-predictions/157.pdf}
        \label{fig:chapter5:predicting-price-157}
    
    }

    \caption[The dynamics of aggregated agents extracted from the simulation.]{\myupdate{Two consecutive} dynamics of aggregated agents extracted from the simulation. The price changes from $p_{i-1}$ to $p_i$ in \myfigref{fig:chapter5:predicting-price-156} and jumps from $p_i$ to $p_{i+1}$ in \myfigref{fig:chapter5:predicting-price-157}. \myfigref{fig:chapter5:predicting-price-156} and \myfigref{fig:chapter5:predicting-price-157} are corresponding to \myfigref{fig:chapter5:predicting-price-157-price} and \myfigref{fig:chapter5:predicting-price-158-price} respectively. \myupdate{The blue curve (dynamics generated by \myupdate{actual scenario}) is the dynamics of aggregated agents that took place in simulation. The orange one (dynamics generated by \myupdate{potential scenario}) is that we want to inspect what the dynamics would be if external agents take the opposite action at the same time steps, i.e., if the blue curve is resulted in by external agents buy the stocks from market, the orange curve is resulted in by the external agents sell the stock to market correspondingly.}}
    \label{fig:chapter5:price-predictions}
\end{figure}

\begin{figure}[ht!]
    \centering

     \subfloat[]{
    \includegraphics[height=4cm,width=\textwidth]{thesis/img/price-predictions/157-price.pdf}
        \label{fig:chapter5:predicting-price-157-price}
    }
    \hfill
    
    \subfloat[]{
    \hbox{\hspace{0.7em}}\includegraphics[height=4cm,width=\textwidth]{thesis/img/price-predictions/158-price.pdf}
            \label{fig:chapter5:predicting-price-158-price}

    }
    \caption[The consecutive predicted prices given by HNN and LSTM networks.]{The consecutive predicted prices given by HNN and LSTM networks. \myfigref{fig:chapter5:predicting-price-156-price} and \myfigref{fig:chapter5:predicting-price-157-price} are corresponding to \myfigref{fig:chapter5:predicting-price-156} and \myfigref{fig:chapter5:predicting-price-157} respectively. \myfigref{fig:chapter5:predicting-price-158-price} is the predicted price after \myfigref{fig:chapter5:predicting-price-157-price}. The blue horizontal line is the ground-truth price, the red horizontal one is the price predicted by LSTM, the light green curve (not the horizontal green line) is the price predicted by HNN 100 times, and the green horizontal line is the average of price predicted by HNN.}
    \label{fig:chapter5:predicting-price-2}
\end{figure}