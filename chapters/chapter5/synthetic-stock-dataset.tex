\newline
\section{Synthetic data sets from financial market model}\label{sec:chapter5:synthetic-data-sets}
% \textbf{Synthetic data sets.}\label{sec:chapter5:synthetic-data-sets}
Following the practical approach described in     \mysectionref{sec:chapter3:practical-approches}, we obtain the following synthetic data sets (see \myfigref{fig:chapter5:market-ground-truth-dataset}). We only display the figures whose standard deviation of $b_n - b_{n-1}$ is 20 since the standard deviation 10 is similar to it. \myupdate{We use the first 1300 points for training set and the rest 400 points for test set.}
\begin{figure}[h!]
    \centering
    \subfloat[]{
        \includegraphics[height=7cm,width=\textwidth]{market-ground-truth-dataset}
    }
    \caption{The synthetic data sets generated from the financial market model. The standard deviation of difference of random walk $b_n - b_{n-1}$ is 20. The top one is the random walk, which is underline in the real market, of the simulation. The middle one is the fluctuation of price based on the model. And the bottom one the change of the total number of stocks in the market, which is following to random walk. }
    \label{fig:chapter5:market-ground-truth-dataset}
\end{figure}
\newline
% \textbf{Setup.} 
\section{Setup}
In the financial market model, we evaluate the LSTM networks with units 64, 128, and 256, and select the best performance. In this thesis, LSTM networks with 64 units \mydelete{beat}\myupdate{outperform} the rest models. As for HNN, we utilize 50 plays and 50 units as hyperparameters to train the network. All these networks are trained with \mydelete{enough}\myupdate{10000} epochs\mydelete{(10k epochs in this thesis)}. For RMSE, we evaluate $d_n = |b_n - b_{n-1}|$ instead of $b_n$ directly due to the loss function given in \mysectionref{sec:chapter4:direct_learning}. To make it clear, we rewrite RMSE as follows,
\begin{equation}
    \text{RMSE} = \sqrt{\frac{\sum_{i=1}^{N}(\hat{d}_i - d_i)^2}{N}}
    % \quad n=0,1,\ldots,N-1
\end{equation}
\newline\newline
% \textbf{Results.} 
\section{Results and analysis}
\mytableref{tbl:chapter5:simulation-stock-results} presents the different performance between LSTM networks and HNN. The data set with $\sigma=20$ means $d_n$ is followed to normal distribution $\mathcal{N}(0, 20)$. We see that HNN outperforms LSTM networks dramatically. HNN can recover the mean $\mu$ and standard deviation $\sigma$ from the random walk accurately, and the results are much more \mydelete{stable}\myupdate{robust} than that learned by LSTM networks \myupdate{comparing the standard deviation of RMSE}. Meanwhile, it reveals that HNN utilizes fewer parameters to produce better RMSE than LSTM.
\myfigref{fig:chapter5:lstm-hnn-stock-mle} shows the curve of $d_n$ generated by LSTM networks and HNN.
We notice that the average of standard deviation obtained by LSTM networks is around 15.24, which is underrated, and \mytableref{tbl:chapter5:simulation-stock-results} confirms this circumstance. Unlike the predicted curve produced by HNN overlapping with the ground-truth one, the blue curve generated by LSTM networks diverges from the ground-truth results and has a smaller magnitude spreading most timesteps. 
Finally \todo{Isn't it for given $(x_n, y_n)$ ? No. It's trained by MLE. The scale of outputs is  not the same} \myfigref{fig:chapter1:market-lstm-result,fig:chapter1:market-hnn-result} present that HNN traces the ground-truth random walk better than LSTM networks do.

\begin{table}[htb!]
\centering
\begin{adjustbox}{angle=0}
\begin{tabular}{||c|c|c|c|c|c||}
\hline 
 Data set & network & \#parameters & estimated $\mu$ & estimated $\sigma$ & RMSE  \\
\hline \hline
\multirow{2}{4em}{$\sigma=10$} & \multirow{1}{4em}{LSTM} & 16961 & 0.05 $\pm$ 0.20 & 5.55 $\pm$ 1.98 & 10.05 $\pm$ 1.39 \\ 
                          \cline{2-6}
                          & \multirow{1}{4em}{HNN}  & 7501 & -0.25  $\pm$ 0.44 & 12.22 $\pm$ 0.38 & 7.40 $\pm$ 0.84 \\ 
                          \cline{2-6}
\hline

\multirow{2}{4em}{$\sigma=20$} & \multirow{1}{4em}{LSTM} & 16961 & 0.05 $\pm$ 0.20 & 15.24 $\pm$ $4.26$ & 18.36 $\pm$ 2.05 \\
                          \cline{2-6}
                          & \multirow{1}{4em}{HNN}  & 7501 & 0.09 $\pm$ 0.20 & 19.74 $\pm$ 0.42  & 8.82 $\pm$ 0.91 \\ 
                          \cline{2-6}
\hline

% \multirow{2}{4em}{$\sigma=110$} & \multirow{1}{4em}{LSTM} & 0 & x & xxxx $\pm$ xxx (not precise)\\ 
%                           \cline{2-5}
%                           & \multirow{1}{4em}{HNN}  & 0 & x & 0 $\pm$ 0 \\ 
%                           \cline{2-5}
% \hline
\end{tabular}
\end{adjustbox}
\caption{RMSE for simulation results on test set} 
\label{tbl:chapter5:simulation-stock-results}
\end{table}
% \FloatBarrier
% \begin{figure}
%     \centering
%     \subfloat[]{
%         \includegraphics[width=\textwidth]{thesis/img/debug-lstm-stock-mle.pdf}
%     }
%     \caption{Caption}
%     \label{fig:chapter5:lstm-stock-mle}
% \end{figure}

\begin{figure}[htb!]
    \centering
    \subfloat[LSTM]{
        \includegraphics[height=3.5cm,width=\textwidth]{thesis/img/lstm-stock-diff-outputs-units-64.pdf}
        \label{fig:chapter5:lstm-stock-mle}
    }
    \hfill
    \subfloat[HNN]{
        \includegraphics[height=3.5cm,width=\textwidth]{thesis/img/hnn-stock-diff-outputs.pdf}
        \label{fig:chapter5:hnn-stock-mle}
    }    
    \caption{Sequence of difference of $b_{n} - b_{n-1}$ on test set. The curve in red is ground-truth and the blue one is generated by trained networks.}
    \label{fig:chapter5:lstm-hnn-stock-mle}
\end{figure} 


% \subsection{analysis-unknown-mu}
% \begin{table}[htb!]
% \centering
% \begin{adjustbox}{angle=0}
% \begin{tabular}{||c|c|c|c|c||}
% \hline 
% network & hyper parameters & \#parameters & RMSE  & predict mu \\
% \hline \hline
% % \multirow{3}{4em}{$\bar{S}$ set} & \multirow{1}{4em}{100} & (25,25) & 0 $\pm$ 0 \\ 
% %                           \cline{2-4}
% %                           & \multirow{1}{4em}{1000}  & (25,25) & 0 $\pm$ 0 \\ 
% %                           \cline{2-4}
% %                           & \multirow{1}{4em}{10000}  & (25,25) & 0 $\pm$ 0 \\ 
% %                           \cline{2-4}
% \hline
% \hline
% \end{tabular}
% \end{adjustbox}
% \caption{RMSE for simulation results}
% \label{tbl:chapter5:simulation-stock-results}
% \end{table}

% \mytodo{add mle loss curve to see that mle in hnn is futhur smaller than lstm}
% \mytodo{add reconstructed agents behaviours graphs}

% \mytodo{show that hnn can reconstruct agents behavior}

\FloatBarrier
% \textbf{Reconstructing \myupdate{aggregated} dynamics of agents.} 
\section{Reconstructing \myupdate{aggregated} dynamics of agents}
\myfigref{fig:chapter5:dynamics-of-agents} displays the dynamics of agents retrieved from the financial market model and HNN. 
The red dotted curve is the dynamics extracted from HNN, and one in orange and blue is the dynamics from the market model. The blue curve is the dynamics of agents taken place if external agents buy (sell) stocks in simulation. Furthermore, the orange one is that we want to inspect what the dynamics if the external agents sell (buy) stocks at the same time steps. \myupdate{The intersection of blue and orange curve is the initial price at that timestep} We stress that the data in orange doesn't show in the training set. \myfigref{fig:chapter5:dynamics-of-agents-1} indicates that HNN can reconstruct the dynamics of agents thoroughly from the training set. 
\myfigref{fig:chapter5:dynamics-of-agents-2} shows the volume of agents decreases slightly, then increases. Similarly, the numeric quantity generated from HNN fits the trajectory. 
The orange curve in \myfigref{fig:chapter5:dynamics-of-agents-3} is unconscious to the training set. However, we observe the red curve produced by HNN reconstructs these dynamics correctly. It implies HNN not only inspects the training set but also restores the agents participated in the market.
Considering \myfigref{fig:chapter5:dynamics-of-agents-4}, we observe the blue curve decreases first, then increases, finally decreases. Meantime the trajectory obtained from HNN traces this tendency. 
Moreover, we can explain that the avalanche of the price based on \myfigref{fig:chapter5:dynamics-of-agents-3,fig:chapter5:dynamics-of-agents-4}. We see there are black and red horizontal lines in \myfigref{fig:chapter5:dynamics-of-agents-3}. If the random walk decreases from -457 to -570 (the black line), then the price only increases from 0.008 to 0.056. However, if the next random walk is a bit lower than -570, take -590 (the red line) for example, we will observe that the price will rise to 0.165, which is triple times than the previous case. That illustrates why some small fluctuations of stocks lead to a large changes in price. 

% Suppose that the change of stocks in market

% The difference between blue and orange curve is that the blue curve is the dynamics of external agents 

\begin{figure}[htb!]
    \centering
    \subfloat[]{
    \includegraphics[width=\textwidth/2]{thesis/img/inspect-agents-behaviours/29.pdf}
        \label{fig:chapter5:dynamics-of-agents-1}

    }
    \subfloat[]{
    \includegraphics[width=\textwidth/2]{thesis/img/inspect-agents-behaviours/31.pdf}
            \label{fig:chapter5:dynamics-of-agents-2}
    }
    \hfill
    \subfloat[]{
    \includegraphics[width=\textwidth/2]{thesis/img/inspect-agents-behaviours/15.pdf}
    \label{fig:chapter5:dynamics-of-agents-3}
    }
    \subfloat[]{
    \includegraphics[width=\textwidth/2]{thesis/img/inspect-agents-behaviours/53.pdf}
    \label{fig:chapter5:dynamics-of-agents-4}
    }    
    \caption{Dynamics of aggregated agents. If the amount of stocks increases, it means the external agents sell stocks to the market, and the price drops. On the contrary, if the amount of stocks decreases, it indicates the external agents buy stocks from the market, and the price rises. 
    \mydelete{The numeric value of the number of agents doesn't matter but the difference between two consecutive time steps, $b_{n} - b_{n-1}$, matters.} \myupdate{The blue curve (dynamics generated by market model I) is the dynamics of agents that took place in simulation. The orange one (dynamics generated by market model II) is that we want to inspect what the dynamics would be if external agents take the opposite action at the same time steps, i.e., if the blue curve is resulted in by external agents buy the stocks from market, the orange curve is resulted in by the external agents sell the stock to market.}}
    \label{fig:chapter5:dynamics-of-agents}
\end{figure}
\FloatBarrier

% \subsection{Prediction}  
% In this evaluation, we only try to predict the trend of next price, up or down, in the next time step. And we also use the following LSTM network architecture to predict the next price.

% \textbf{LSTM network architecture.} 


% In order the compare the results of LSTM
% \mytodo{show that hnn can give a distribution of price, not average of price} 
