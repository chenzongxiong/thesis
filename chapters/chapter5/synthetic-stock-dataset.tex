\newline
\section{Synthetic data sets from financial market model}\label{sec:chapter5:synthetic-data-sets}
Following the practical approach described in     \mysectionref{sec:chapter3:practical-approches}, we obtain the following synthetic data sets (see \myfigref{fig:chapter5:market-ground-truth-dataset}). We only display the figures whose standard deviation of $b_n - b_{n-1}$ is 20 since the standard deviation 10 is similar to it. \myupdate{We use the first 1300 points for training set and the rest 400 points for test set.}
\begin{figure}[h!]
    \centering
    \subfloat[]{
        \includegraphics[height=7cm,width=\textwidth]{market-ground-truth-dataset}
    }
    \caption[The synthetic data sets generated from the financial market model.]{The synthetic data sets generated from the financial market model. The standard deviation of the difference of the random walk $b_n - b_{n-1}$ is 20. The top one is the random walk, which is underline in the real market, of the simulation. The middle one is the fluctuation of the prices based on the model. And the bottom one the change of the total number of stocks in the market, which is following to random walk.}
    \label{fig:chapter5:market-ground-truth-dataset}
\end{figure}
\newline
% \textbf{Setup.} 
\section{Setup}
In the financial market model, we evaluate the LSTM networks with units 64, 128, and 256, and select the best performance. In this thesis, LSTM networks with 64 units \myupdate{outperform} the rest models. As for HNN, we utilize 50 \textit{\_\_plays\_\_} and 50 \textit{\_\_units\_\_} to train the network. All these networks are trained with \myupdate{10000} epochs. For RMSE, we evaluate $d_n = |b_n - b_{n-1}|$ instead of $b_n$ directly since the sequence $\mathcal{B}_n = \{b_0, b_1, \ldots, b_n\}$ is unobservable to us (see \mysectionref{sec:chapter4:assumption}). \myupdate{One should analyze the prices on test set and $d_n$ are mostly interesting.} To make it clear, we rewrite RMSE as follows,
\begin{eqnarray*}
    \text{RMSE} &= \sqrt{\frac{\sum_{i=1}^{N}(\hat{d}_i - d_i)^2}{N}} \\
    s.t. & \hat{d}_i = \hat{b}_i - \hat{b}_{i-1} \\
    & d_i = b_i - b_{i-1}
\end{eqnarray*}
where $b_i$ is the underline ground-truth noise (extracted from financial market model) and $\hat{b}_i$ is the underline predictive noise (reconstructed by HNN).

% \textbf{Results.} 
\section{Results and analysis}
\mytableref{tbl:chapter5:simulation-stock-results} presents the different performance between LSTM networks and HNN. The data set with $\sigma=20$ means $d_n$ is followed to normal distribution $\mathcal{N}(0, 20)$. We see that HNN outperforms LSTM networks dramatically. HNN can recover the mean $\mu$ and standard deviation $\sigma$ from the random walk accurately, and the results are much more \myupdate{robust} than that learned by LSTM networks \myupdate{comparing the standard deviation of RMSE}. Meanwhile, it reveals that HNN utilizes fewer parameters to produce better RMSE than LSTM.
\myfigref{fig:chapter5:lstm-hnn-stock-mle} shows the curve of $d_n$ generated by LSTM networks and HNN.
We notice that the average of standard deviation obtained by LSTM networks is around 15.24, which is underrated, and \mytableref{tbl:chapter5:simulation-stock-results} confirms this circumstance. Unlike the predicted curve produced by HNN overlapping with the ground-truth one, the blue curve generated by LSTM networks diverges from the ground-truth results and has a smaller magnitude spreading most time steps. 
\myfigref{fig:chapter6:market-lstm-result,fig:chapter6:market-hnn-result} present that HNN traces the ground-truth random walk better than LSTM networks do. In  \myfigref{fig:chapter6:market-lstm-result,fig:chapter6:market-hnn-result}, the scale of their outputs are not the same since both HNN and LSTM networks do not learn the standard deviation precisely.
We present a more detailed comparison of random walk generated by HNN and LSTM networks in \myappendixsectionref{appendix:comparsion-random-walk}

\begin{table}[htb!]
\centering
\begin{adjustbox}{angle=0}
\begin{tabular}{||c|c|c|c|c|c||}
\hline 
 Data set & network & \#parameters & estimated $\mu$ & estimated $\sigma$ & RMSE  \\
\hline \hline
\multirow{2}{4em}{$\sigma=10$} & \multirow{1}{4em}{LSTM} & 16961 & 0.05 $\pm$ 0.20 & 5.55 $\pm$ 1.98 & 10.05 $\pm$ 1.39 \\ 
                          \cline{2-6}
                          & \multirow{1}{4em}{HNN}  & 7501 & -0.25  $\pm$ 0.44 & 12.22 $\pm$ 0.38 & 7.40 $\pm$ 0.84 \\ 
                          \cline{2-6}
\hline

\multirow{2}{4em}{$\sigma=20$} & \multirow{1}{4em}{LSTM} & 16961 & 0.05 $\pm$ 0.20 & 15.24 $\pm$ $4.26$ & 18.36 $\pm$ 2.05 \\
                          \cline{2-6}
                          & \multirow{1}{4em}{HNN}  & 7501 & 0.09 $\pm$ 0.20 & 19.74 $\pm$ 0.42  & 8.82 $\pm$ 0.91 \\ 
                          \cline{2-6}
\hline


\end{tabular}
\end{adjustbox}
\caption{RMSE for simulation results on the test set} 
\label{tbl:chapter5:simulation-stock-results}
\end{table}


\begin{figure}[htb!]
    \centering
    \subfloat[LSTM]{
        \includegraphics[height=2.8cm,width=\textwidth]{thesis/img/lstm-stock-diff-outputs-units-64.pdf}
        \label{fig:chapter5:lstm-stock-mle}
    }
    \hfill
    \subfloat[HNN]{
        \includegraphics[height=2.8cm,width=\textwidth]{thesis/img/hnn-stock-diff-outputs.pdf}
        \label{fig:chapter5:hnn-stock-mle}
    }    
    \caption[Sequence of difference of $b_{n} - b_{n-1}$ on the test set.]{Sequence of difference of $b_{n} - b_{n-1}$ on the test set. The curve in red is ground-truth and the blue one is generated by trained networks.}
    \label{fig:chapter5:lstm-hnn-stock-mle}
\end{figure} 
%% TODO'
\begin{figure}[htb!]
    \centering
    \subfloat[LSTM]{
        \scalebox{1.0} {
        \includegraphics[height=2.8cm,width=\textwidth]
    % {img/market-prediction-lstm/6.pdf}
     {img/market-prediction-lstm/1.pdf}
        }
        \label{fig:chapter6:market-lstm-result}
    }
    \hfill
    \subfloat[HNN]{
        \scalebox{1.0} {
        \includegraphics[height=2.8cm,width=\textwidth]
        % {img/market-prediction-hnn/6.pdf}
         {img/market-prediction-hnn/1.pdf}
        }
        \label{fig:chapter6:market-hnn-result}
    }
    \caption[The predictive outputs against ground-truth outputs for LSTM and HNN.]{The predictive against ground-truth outputs on test set. The red curve is ground-truth and the blue one is generated by trained networks}
    \label{fig:chapter6:market-hnn-lstm-results}
\end{figure}


\FloatBarrier
% \textbf{Reconstructing \myupdate{aggregated} dynamics of agents.} 
\section{Reconstructing \myupdate{aggregated} dynamics of agents}
\myfigref{fig:chapter5:dynamics-of-agents} displays the dynamics of aggregated agents retrieved from the financial market model and HNN. 
The red dotted curve is the dynamics extracted from HNN, and one in orange and blue is the dynamics from the market model. The blue curve is the dynamics of agents taken place if external agents buy (sell) stocks in simulation. On the contrary, the orange one is that we want to inspect what the dynamics if the external agents sell (buy) stocks at the same time step. \myupdate{The intersection of blue and orange curve is the initial price at that time step.} We stress that the data in orange are not presented in the training set \myupdate{and we will see that HNN can reconstruct the orange curves properly without observing them in training set}. \myfigref{fig:chapter5:dynamics-of-agents-1} indicates that HNN can reconstruct the dynamics of aggregated agents thoroughly from the training set. 
\myfigref{fig:chapter5:dynamics-of-agents-2} shows that the volume of aggregated agents decreases slightly, then increases. Similarly, the numeric quantity generated from HNN fits the trajectory. 
The orange curve in \myfigref{fig:chapter5:dynamics-of-agents-3} is unconscious to \myupdate{the HNN due to the absence in} training set. However, we observe the red curve produced by HNN reconstructs these dynamics correctly. It implies HNN not only inspects the training set but also restores \myupdate{the strategies of} the agents participated in the market.
Considering \myfigref{fig:chapter5:dynamics-of-agents-4}, we observe the blue curve decreases first, then increases, finally decreases. Meantime the trajectory obtained from HNN follows this tendency. 
Moreover, we can explain that the avalanche of the \myupdate{price} based on \myfigref{fig:chapter5:dynamics-of-agents-3,fig:chapter5:dynamics-of-agents-4}. We see there are black and red horizontal lines in \myfigref{fig:chapter5:dynamics-of-agents-3}. If the random walk decreases from -457 (the intersection of orange and blue curve) to -570 (the black horizontal line), then the price only increases from 0.008 to 0.056. However, if the next random walk is a bit lower than -570, take -590 (the red horizontal line) for example, we will observe that the price will rise to 0.165, which is triple times than the previous case. That illustrates why some small fluctuations of stocks lead to a large \myupdate{changes} in price. It coincides to the analysis in \myassumptionref{assumption:chapter3:transitions-between-stabilized-prices} and \myfigref{fig:chapter3:price-change-bifurcation-2}.

% Suppose that the change of stocks in market

% The difference between blue and orange curve is that the blue curve is the dynamics of external agents 

\begin{figure}[htb!]
    \centering
    \subfloat[]{
    \includegraphics[width=\textwidth/2]{thesis/img/inspect-agents-behaviours/29.pdf}
        \label{fig:chapter5:dynamics-of-agents-1}
    }
    \subfloat[]{
    \includegraphics[width=\textwidth/2]{thesis/img/inspect-agents-behaviours/31.pdf}
            \label{fig:chapter5:dynamics-of-agents-2}
    }
    \hfill
    \subfloat[]{
    \includegraphics[width=\textwidth/2]{thesis/img/inspect-agents-behaviours/15.pdf}
    \label{fig:chapter5:dynamics-of-agents-3}
    }
    \subfloat[]{
    \includegraphics[width=\textwidth/2]{thesis/img/inspect-agents-behaviours/53.pdf}
    \label{fig:chapter5:dynamics-of-agents-4}
    }    
    \caption[Dynamics of aggregated agents.]{Dynamics of aggregated agents \myupdate{corresponding to four different independent events} on the market on test set. If the amount of stocks increases, it means the external agents sell stocks to the market, and the price drops. On the contrary, if the amount of stocks decreases, it indicates the external agents buy stocks from the market, and the price rises. 
    \myupdate{The blue curve (dynamics generated by \myupdate{actual scenario}) is the dynamics of aggregated agents that took place in the simulation. The orange one (dynamics generated by \myupdate{potential scenario}) is that we want to inspect what the dynamics would be if external agents take the opposite action at the same time steps, i.e., if the blue curve is resulted in by external agents buy the stocks from market, the orange curve is resulted in by the external agents sell the stock to market correspondingly.}}
    \label{fig:chapter5:dynamics-of-agents}
\end{figure}
\FloatBarrier
