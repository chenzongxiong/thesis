\subsection{Demand/Supply price formation\label{sec:chapter3:demand-supply-price-formation}}
In this model, we assume that the stock exchange accommodates two types of agents, which we call D and N and describe below. We assume that the time takes values $n=0, 1, 2, , \ldots$, and denote the price at time $n$ by $p_n$. By defnition, $p_n$ is the price of last transaction at time $n$. We will see below that demand equals supply at times $n$, but between any two consecutive times $n-1$ and $n$, this need not be the case and there may occur many transactions until demand will have become supply.

\begin{assumption}\label{assumption:chapter3:strategy-of-agentD}
  (Strategy of agents D). Agents D keep track of a trend. They buy stocks iff the price goes up and sell stocks iff the pric goes down. The total amount of stocks that are in possession of all the agents D can be described as a Prandtl-Ishlinksii operator whose input is the price p. We denote this operator by
  $$P_D(p)$$
\end{assuption}
\begin{assumption}\label{assumption:chapter3:strategy-of-agentN}
  (Strategy of agents N). The strategy of each of the agents N is characterized by a non-ideal relay with two fixed threshold $p_1 < p_2$ (different for different agents). The relay is in state 0 if the price higher than $p_2$ and in state 1 if the price is lower than $p_1$. The agent buys one stock whenever his relay switches from 0 to 1 and sells one stock whenever his relay switches frmo 1 to 0. The total amount of stocks that are in possession of all the agents N can be described as a Preisach operator whose input is the price $p$. We denote this operator by
  $$P_N(p)$$
\end{assumption}

The following lemma is direct consequence of the definition of the opeartors $P_D(p)$ and $P_N(p)$.

\begin{lemma}\label{lemma:chapter3:reaction-to-price-change}
  \\
  \begin{enumerate}
    \item If $p$ is increasing, the $P_D(p)$ is increasing (agents D buy) and $P_N(p)$ is descreasing (agents N sell).
    \item If $p$ is descreasing, the $P_D(p)$ is descreasing (agents D sell) and $P_N(p)$ is increasing (agents N buy).
  \end{enumerate}
\end{lemma}

\begin{remark}\label{remark:chapter3:1}
  Since $P_D(p)$ and $P_N(p)$ are not functions but operators, the monotone curves described in \ref{lemma:chapter3:reaction-to-price-change} are not defined only by the current value of $p$, but depend on its prehistory.
\end{remark}

Denote by $B_n$ the total amount of stocks that are in possession of all the agents D and N at time $n$. We will see below in \ref{assumption:chapter3:transitions-between-stabilized-prices}

\begin{assumption}\label{assumption:chapter3:demand/supply-price-formation}
  At each moment $n$, the price $p_n$ is a stable solution of equation
  \begin{equation}\label{eqn:chapter3:demand-supply-price-formation}
    P_D(p_n) + P_N(p_n) = B_n
  \end{equation}
  where the stability notion is explained in Fig.\ref{xxxx}, We will say that $p_n$ is a stabilized price.
\end{assumption}

In order to explain how a stabilized price can change stabilize to the next value, we make the next assumption.

\begin{assumption}\label{assumption:chapter3:transitions-between-stabilized-prices}
  Suppose $p_{n-1}$ is a stabilized price as time $n-1$. In particular, it is a stable solution of equation
  \begin{equation}\label{eqn:chapter3:stable-price-eqn}
    P_D(p_{n-1}) + P_N(p_{n-1}) = B_{n-1}
  \end{equation}
  We assue that between time moments $n-1$ and $n$, some external agents E buy or sell some stocks. As a result, the total amount of stocks that is in possession of agents D and N becomes $B_n$. Moreover, we assume that external agents do not stay at the stock exchange, i.e., the operators (thir densities) $P_D(p)$ and $P_N(p)$ do not chagne in time.
  If $B_n \le B_{n-1}$, then external agents E buy $|B_{n}-B_{n-1}|$ stocks, which increases the price. As as result, agents D also buy. All the stocks bought by agents E and D are sold by agnets N. This leads to the increase of the price to a new value $p_n^1$, where $p_n^1$ is the smallest solution of the equation
  \begin{equation}\label{eqn:chapter3:new-stable-price-eqn}
    P_D(p_n^1) + P_N(p_n^1) = B_n
  \end{equation}
  satisfying the inequality
  \begin{equation}
    p_n^1 \ge p_{n-1}
  \end{equation}

  Note that small values $|B_n - B_{n-1}|$ may correspond to small change of the price as in Fig.\ref{xxxx} or large changes as in Fig.\ref{xxxx}. In the latter case, the price jumps up due to the fold bifurcation.

  If $p_n^1$ is a stable of equation \ref{eqn:chapter3:new-stable-price-eqn}, the, by definition, it coincides with the new stabilized price:
  \begin{equation}\label{eqn:new-price}
    p_n := p_n^1
  \end{equation}
  Otherwise, the price makes several jumps taking semi-stable values $p_n^2, \ldots, p_n^{m-1}$ and a stable value $p_n^m$ (\textbf{hopefully with a finite} m) such that
  \begin{equation}
    missing now
  \end{equation}
  By definition, we set
  \begin{equation}
    p_n := p_n^m
  \end{equation}
  Analogously, the price will leave the stable value $p_{n-1}$ if $B_n > B_{n-1}$. In this case external agents E sell $B_n - B_{n-1}$ stocks, which decreases the price. As a result, agents D also sell. All the stocks sold by agents E and D are bought by agent N.
\end{assumption}

  The last assumption concerns the strategy of external agents E.
  \begin{assumption}
    $B_n$ is a Markov chain. For example, $B_n \sim \mathcal{N}(B_{n-1} + \mu, \tau^{-1})$ with some mean $\mu$ and precision $\tau > 0$
  \end{assumption}

  \begin{remark}\label{remark:chapter3:TODO}
    Set
    \begin{equation}
      G(p) := P_D(p) + P_N(p)
    \end{equation}
  \end{remark}

  Then, formally, the relationship between the price $p_n$ and the noise $B_n$ is the same as Dima's model. \ref{xxx} Though in our second model, there is a number of further restrictions on admissible vlaues of $p_n$ due to Assumption \ref{assumption:chapter3:transitions-between-stabilized-prices}
