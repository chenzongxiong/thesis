\section{Hysteretic neural networks\label{sec:chapter3:hnn}}
\subsection{Play and Prandtl-Ishlinskii networks\label{sec:chapter3:play-and-pi-networks}}

Consider $K > 0$ play operators. Each of them maps an initial state $p_{0}^{k} \in \mathbb{R} $ and an input sequence $x_1, x_2, \ldots$ to an output sequence $p_{1}^{k}, p_{2}^{k}, \ldots \ $, i.e.,

% \begin{equation}\label{eqn:input_to_op_output_mapping}
\begin{equation*}
  p_{0}^{k}, (x_1, x_2, \ldots) \mapsto (p_{1}^{k}, p_{2}^{k}, \ldots), k = 1, \ldots, K
\end{equation*}

The $k$th play operator is given by:

\begin{equation}\label{eqn:chapter3:play-operator}
  p_{n}^{k} = G(x_{n}, p_{n-1}^{k}, w^{k}) := p_{n-1}^{k} + \Phi(w^{k} x_{n} - p_{n-1}^{k}), n = 1, 2, \ldots, N
\end{equation}

where $w^{k}$ are parameters and

\begin{equation}\label{eqn:chapter3:phi}
  \begin{aligned}
    \Phi(x) =
    \begin{cases}
      x - \frac{1}{2}, & x > \frac{1}{2} \\
      0,               & -\frac{1}{2} <= x <= \frac{1}{2} \\
      x + \frac{1}{2}, & x < \frac{1}{2}
    \end{cases}
  \end{aligned}
\end{equation}

See Fig. \ref{fig:chapter3:phi}

\documentclass{standalone}
\usepackage{tikz}
\begin{document}
\begin{tikzpicture}
% % horizontal axis
% \draw[->] (0,0) -- (6,0) node[anchor=north] {$f/f_N$};
% % labels
% \draw	(0,0) node[anchor=north] {0}
% 		(2,0) node[anchor=north] {1}
% 		(4,0) node[anchor=north] {2};
% % ranges
% \draw	(1,3.5) node{{\scriptsize Constant flux}}
% 		(4,3.5) node{{\scriptsize Field weakening}};

% % vertical axis
% \draw[->] (0,0) -- (0,4) node[anchor=east] {$U_s,\varPsi_s$};
% % nominal speed
% \draw[dotted] (2,0) -- (2,4);

% % Us
% \draw[thick] (0,0) -- (2,2) -- (6,2);
% \draw (1,1.5) node {$U_s$}; %label

% % Psis
% \draw[thick,dashed] (0,3) -- (2,3) parabola[bend at end] (6,1);
% \draw (2.5,3) node {$\varPsi_s$}; %label

  \begin{axis} [
    xmin=-2.5,xmax=2.5,
    ymin=-2,ymax=2,
    grid=both,
    ylable={$\Phi(x)$}, xlabel={$x$},
    xtick={-2,-1.5,...,2},
    ytick={-2,-1.5,...,2},
    xticklabel style = {font=\tiny, xshift=0.5ex},
    yticklabel style = {font=\tiny, yshift=0.5ex},
    axis line style={->},
    axis x line=middle,
    axis y line=middle
    ]
    \addplot+[mark=none, black, domain=-2:-0.5] {x+0.5};
    \addplot+[mark=onne, black, domain=-0.5:0.5] {0};
    \addplot+[mark=none, black, domain=0.5:2] {x-0.5};
    % \addlegendentry{$\Phi(x)$}
  \end{axis}
% \draw [<->,thick] (-2,2) node (yaxis) [above] {$\Phi(x)$}
%    |- (3,0) node (xaxis) [right] {$x$}
% % axis
% \draw [->] (-2, 0) -- coordinate (x axis mid) (2, 0);
% \draw [->] (0, -2) -- coordinate (y axis mid) (0, 2);
% % ticks
% \foreach \x in {-2,-1.5,...,2}
% \draw (\x,1pt) -- (\x,-3pt)
% node[archor=north] {\x};
% \foreach \y in {-2,-1.5,...,2}
% \draw (1pt,\y) -- (-3pt,\y)
% node(anchor=east) {\y};

% \draw (-2,-1.5) -- (-0.5,0);
% \draw (-0.5,0) -- (0.5,0);
% \draw (0.5,0) -- (2,1.5);

\end{tikzpicture}
\end{document}

\input{./tikz/chapter3/arch}

% \documentclass{standalone}
\usepackage{tikz}
\begin{document}
\begin{tikzpicture}
% % horizontal axis
% \draw[->] (0,0) -- (6,0) node[anchor=north] {$f/f_N$};
% % labels
% \draw	(0,0) node[anchor=north] {0}
% 		(2,0) node[anchor=north] {1}
% 		(4,0) node[anchor=north] {2};
% % ranges
% \draw	(1,3.5) node{{\scriptsize Constant flux}}
% 		(4,3.5) node{{\scriptsize Field weakening}};

% % vertical axis
% \draw[->] (0,0) -- (0,4) node[anchor=east] {$U_s,\varPsi_s$};
% % nominal speed
% \draw[dotted] (2,0) -- (2,4);

% % Us
% \draw[thick] (0,0) -- (2,2) -- (6,2);
% \draw (1,1.5) node {$U_s$}; %label

% % Psis
% \draw[thick,dashed] (0,3) -- (2,3) parabola[bend at end] (6,1);
% \draw (2.5,3) node {$\varPsi_s$}; %label

  \begin{axis} [
    xmin=-2.5,xmax=2.5,
    ymin=-2,ymax=2,
    grid=both,
    ylable={$\Phi(x)$}, xlabel={$x$},
    xtick={-2,-1.5,...,2},
    ytick={-2,-1.5,...,2},
    xticklabel style = {font=\tiny, xshift=0.5ex},
    yticklabel style = {font=\tiny, yshift=0.5ex},
    axis line style={->},
    axis x line=middle,
    axis y line=middle
    ]
    \addplot+[mark=none, black, domain=-2:-0.5] {x+0.5};
    \addplot+[mark=onne, black, domain=-0.5:0.5] {0};
    \addplot+[mark=none, black, domain=0.5:2] {x-0.5};
    % \addlegendentry{$\Phi(x)$}
  \end{axis}
% \draw [<->,thick] (-2,2) node (yaxis) [above] {$\Phi(x)$}
%    |- (3,0) node (xaxis) [right] {$x$}
% % axis
% \draw [->] (-2, 0) -- coordinate (x axis mid) (2, 0);
% \draw [->] (0, -2) -- coordinate (y axis mid) (0, 2);
% % ticks
% \foreach \x in {-2,-1.5,...,2}
% \draw (\x,1pt) -- (\x,-3pt)
% node[archor=north] {\x};
% \foreach \y in {-2,-1.5,...,2}
% \draw (1pt,\y) -- (-3pt,\y)
% node(anchor=east) {\y};

% \draw (-2,-1.5) -- (-0.5,0);
% \draw (-0.5,0) -- (0.5,0);
% \draw (0.5,0) -- (2,1.5);

\end{tikzpicture}
\end{document}

%\begin{figure}[htb]
%   \centering
   % \resizebox{8cm}{!}{\documentclass{standalone}
\usepackage{tikz}
\begin{document}
\begin{tikzpicture}
% % horizontal axis
% \draw[->] (0,0) -- (6,0) node[anchor=north] {$f/f_N$};
% % labels
% \draw	(0,0) node[anchor=north] {0}
% 		(2,0) node[anchor=north] {1}
% 		(4,0) node[anchor=north] {2};
% % ranges
% \draw	(1,3.5) node{{\scriptsize Constant flux}}
% 		(4,3.5) node{{\scriptsize Field weakening}};

% % vertical axis
% \draw[->] (0,0) -- (0,4) node[anchor=east] {$U_s,\varPsi_s$};
% % nominal speed
% \draw[dotted] (2,0) -- (2,4);

% % Us
% \draw[thick] (0,0) -- (2,2) -- (6,2);
% \draw (1,1.5) node {$U_s$}; %label

% % Psis
% \draw[thick,dashed] (0,3) -- (2,3) parabola[bend at end] (6,1);
% \draw (2.5,3) node {$\varPsi_s$}; %label

  \begin{axis} [
    xmin=-2.5,xmax=2.5,
    ymin=-2,ymax=2,
    grid=both,
    ylable={$\Phi(x)$}, xlabel={$x$},
    xtick={-2,-1.5,...,2},
    ytick={-2,-1.5,...,2},
    xticklabel style = {font=\tiny, xshift=0.5ex},
    yticklabel style = {font=\tiny, yshift=0.5ex},
    axis line style={->},
    axis x line=middle,
    axis y line=middle
    ]
    \addplot+[mark=none, black, domain=-2:-0.5] {x+0.5};
    \addplot+[mark=onne, black, domain=-0.5:0.5] {0};
    \addplot+[mark=none, black, domain=0.5:2] {x-0.5};
    % \addlegendentry{$\Phi(x)$}
  \end{axis}
% \draw [<->,thick] (-2,2) node (yaxis) [above] {$\Phi(x)$}
%    |- (3,0) node (xaxis) [right] {$x$}
% % axis
% \draw [->] (-2, 0) -- coordinate (x axis mid) (2, 0);
% \draw [->] (0, -2) -- coordinate (y axis mid) (0, 2);
% % ticks
% \foreach \x in {-2,-1.5,...,2}
% \draw (\x,1pt) -- (\x,-3pt)
% node[archor=north] {\x};
% \foreach \y in {-2,-1.5,...,2}
% \draw (1pt,\y) -- (-3pt,\y)
% node(anchor=east) {\y};

% \draw (-2,-1.5) -- (-0.5,0);
% \draw (-0.5,0) -- (0.5,0);
% \draw (0.5,0) -- (2,1.5);

\end{tikzpicture}
\end{document}
}
%    \input{./tikz/arch}
   % \caption{$\Phi(x)$}\label{fig:chapter3:phi}
%\end{figure}

It can be represented as a recurrent neural network, see Fig. \ref{fig:chapter3:phi}. Note that in such a form the network is not feed-forward.
One can unfold it to make it feed-forward, see Fig. \ref{fig:chapater3:play-operator}

\begin{definition}
We call this network a \textsl{play network}. If there are \textsl{m} elements in the sequence ${x_n}$, we say the unfolded network is \textsl{m-unfolded}
\end{definition}

For example, the network in Fig. \ref{fig:chapter3:unfolded-nn} is 2-unfolded.

\begin{figure}[htb]
    \centering
    \input{./tikz/chapter3/arch}
    \caption{arch}
\end{figure}