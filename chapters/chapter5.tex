\chapter{Evaluation\label{cha:chapter5}}

% \section{Attribute to compare}\label{}
% \mytodo{complexity of synthesis data set (nb plays/nb units/random weights/noise)} 
% \mytodo{number of sequence length (points)} 
% \mytodo{number of parameters to use in the network, our network is smaller and can  achieve the same performance as (lstm/rnn/gru)} 
% \mytodo{training loss curve} 
% \mytodo{time consumption during train/prediction} 
% \mytodo{(mean square error/cross entropy) on prediction data set}
% In this chapter, we evaluate different methods on hysteretic data sets and compare the performance among them.

\section{Data sets}
We generate data sets to identify whether LSTM/SimpleRNN/GRU network and HNN could capture the micro-loops of a hysteretic process. 
\newline
\textbf{Data set $D$.} The set $D$ (see \myfigref{fig:chapter5:dima-seq} and \myfigref{fig:chapter5:dima-seq-2}) consists 1000 points. We repeat sequence $0, 3, 5, 0, 1, 5$ to generate the input $x(n)$ for $D$. We also insert $0, -100$ at the very beginning of the input $x(n)$ to erase the memory in the hysteretic process. The first 600 points are used as a training set, and the rest 400 points are for the test set. We rescale the input of test set by 1/10, 1/7, 1/6, 1/5, 1/4, 1/3, 1/1, 1/0.5 every 50 points to identify inner loops. In this artificial set, only customized hysteretic loops are exposed to networks throughout the training phase.
% The micro and macro loops in test set  any model during learning. 
\newline
\textbf{Data set $P$.} For the set $P$ (see \myfigref{fig:chapter5:pavel-seq} and \myfigref{fig:chapter5:pavel-seq-2}), it also contains 1000 points. The inputs $x(n)$ is sampled from a periodical function $5cos(0.1n)$ with random noise $noise$. Repeatedly, $noise$ is drawn from different normal distributions, which standard deviations are $0.1, 0.5, 1, 2, 3, 4, 5$. \todo{why do I use this data set}
\newline
\textbf{Data set $S$.} We generate set $S$ (see \myfigref{fig:chapter5:sin-seq} and \myfigref{fig:chapter5:sin-seq-2}) with different number of points. The inputs $x(n)$ of $S$ is sampled from a periodical signal function $\cos(0.1n) + 0.3\sin(1.3n) + 1.2\sin(0.6n)$. 
\newline
% \textbf{Data set $M$} \mytodo{the outputs is a random walk}

% \begin{figure}[h!]
%     \centering
%     \subfloat[]{
%     \includegraphics[height=7cm,width=\textwidth]{thesis/img/debug-input-output-dima.pdf}
%     }
%     \caption{Data set D. The first and second plots are inputs vs. time step $n$ and outputs vs. time step $n$, respectively. Time step $n$ is between 10 and 1000. The last one is outputs vs. inputs, only showing $y(n) \in [-0.5, 45]$}
%     \label{fig:chapter5:dima-seq}
% \end{figure}

% \begin{figure}[h!]
%     \centering
%     \includegraphics[height=7cm,width=\textwidth]{thesis/img/debug-input-output-pavel.pdf}
%     \caption{Data set P. The first and second plots are inputs vs. time step $n$ and outputs vs. time step $n$, respectively. Time step $n$ is from 0 to 1000. The last one is outputs vs. inputs, only showing $y(n) \in [-0.5, 45]$}
%     \label{fig:chapter5:pavel-seq}
% \end{figure}

% \begin{figure}[h!]
%     \centering
%     \subfloat[]{
%         \includegraphics[height=5cm,width=\textwidth]{thesis/img/debug-input-output-sin-2.pdf}
%     }
%     \hfill
%     \subfloat[]{
%         \includegraphics[height=5cm,width=10cm]{thesis/img/debug-input-output-sin.pdf}
%     }
%     \caption{Data set S. The first and second plots are inputs vs. time step $n$ and outputs vs. time step $n$, respectively. Time step $n$ is from 0 to 1000.}
%     \label{fig:chapter5:s-seq}
% \end{figure}
\begin{figure}[ht!]
    \centering

    \subfloat[]{
        \includegraphics[width=\textwidth/3]{debug-input-output-dima.pdf}
                 \label{fig:chapter5:dima-seq-2}

    }
    \subfloat[]{
        \includegraphics[width=\textwidth/3]{debug-input-output-pavel.pdf}
                         \label{fig:chapter5:pavel-seq-2}

    }
    \subfloat[]{
        \includegraphics[width=\textwidth/3]{debug-input-output-sin.pdf}
                         \label{fig:chapter5:sin-seq-2}

    }

    \hfill
    
    \caption{Caption}
    \label{fig:chapter5:data-sets}
\end{figure}

\begin{figure}[ht!]
    \centering
    \subfloat[]{
        \includegraphics[height=3cm,width=\textwidth]{debug-input-output-dima-2.pdf}
         \label{fig:chapter5:dima-seq}
    }
    
    \hfill
    \subfloat[]{
        \includegraphics[height=3cm,width=\textwidth]{debug-input-output-pavel-2.pdf}
                 \label{fig:chapter5:pavel-seq}

    }
    \hfill
    \subfloat[]{
        \includegraphics[height=3cm,width=\textwidth]{debug-input-output-sin-2.pdf}
                 \label{fig:chapter5:sin-seq}

    }
    \caption{Data sets.}
    \label{fig:chapter5:data-sets}
\end{figure}
\FloatBarrier

\section{Architectures}\label{sec:chapter5:architectures}

\textbf{Hyperparameters}
In this evaluation, we focus on optimizing the following hyper parameters in LSTM/SimpleRNN/GRU networks and HNN.
\begin{table}[h!]
\begin{center}
    \begin{tabular}{||c|c||}
    \hline
    Name                  & Explanation \\
    \hline 
    \#nb\_plays           & the number of $plays$ in ground-truth data sets \\
    \hline
    \#units               & the number of $units$ in ground-truth data sets \\
    \hline
     \#\_\_nb\_plays\_\_  & the number of $plays$ used in training phase \\
     \hline
     \#\_\_units\_\_      & the number of $units$ used in training phase \\
    \hline
    \end{tabular}
    \caption{Hyperparameters used for evaluation. \#\_\_nb\_plays\_\_ and \#\_\_units\_\_  can be different from the ground-truth parameters since they're blind to us in reality.}
    \label{tbl:chapter5:hyperparameters}    
\end{center}
\end{table}

\textbf{LSTM network.} We use one LSTM layer with tanh nonlinearities and one dense layer. We use the default settings of hyper parameters in tensorflow \citep{abadi2016tensorflow} except for the hidden units in LSTM layer. We grid search a sequence of hidden units, $1, 8, 16, 32, 64, 128, 256$, to find out the best results achieved by LSTM networks. As for the loss functions, we minimize mean squared error (MSE) and minimize the negative maximal likelihood estimator (MLE). The total number of parameters used in this architecture is given by
\begin{equation*}
    \text{\#parameters} = 4 * ({\text{\#\_\_units\_\_}}^2 + 2*\text{\#\_\_units\_\_}) + \text{\#\_\_units\_\_} + 1
\end{equation*}
\textbf{SimpleRNN/GRU network.} As for SimpleRNN/GRU network, we use the same architecture as LSTM network except the recurrent block. In SimpleRNN/GRU network, we apply SimpleRNN/GRU block instead of LSTM block. The total number of parameters used in these two different networks are given by 

\begin{equation*}
   \text{\#parameters} = ({\text{\#\_\_units\_\_}}^2 + 2*\text{\#\_\_units\_\_}) + \text{\#\_\_units\_\_} + 1    
\end{equation*}
and
\begin{equation*}
   \text{\#parameters} = 3 * ({\text{\#\_\_units\_\_}}^2 + 2*\text{\#\_\_units\_\_}) + \text{\#\_\_units\_\_} + 1    
% \text{\#parameters} = 3 * \text{\#\_\_units\_\_} + 3 * \text{\#\_\_units\_\_} * \text{\#\_\_units\_\_} + 3 * \text{\#\_\_units\_\_} + \text{\#\_\_units\_\_} + 1
\end{equation*}
 respectively.


% where $\#units$ is the number of hidden units.

\textbf{HNN.}  We use architecture shown in \mychapterref{cha:chapter1} (see \myfigref{fig:chapter1:nn-arch}). We also exploit hyperparameters $\_\_plays\_\_$, $\_\_units\_\_$ in HNN. The number of parameters $\#parameters$ used in HNN is given by
\begin{equation*}
    \text{\#parameters} = 3 * \text{\#\_\_plays\_\_} * \text{\#\_\_units\_\_} + 1
\end{equation*}
% where $\#play$ is the number of plays and $\#units$ is the number of units.


\section{Measure}
The overall \textit{root mean squared error} (RMSE) is used to measure the quality of fit. It's defined as following,
% \begin{equation}
% \text{RMSE} = \sqrt{\frac{\sum_{i=0}^{N-1}(\hat{y}_i - y_i)^2}{N}}
% \end{equation}
% Additionally, we defined RMSE(n) as
\begin{equation}
    \text{RMSE(n)} = \sqrt{\frac{\sum_{i=n}^{N-1}(\hat{y}_i - y_i)^2}{N-n}}, \quad n=0,1,\ldots,N-1
\end{equation}
where $n$ means the first $(n-1)$th data points are ignored in RMSE results.

Particularly, we set $\text{RMSE}=\text{RMSE(0)}$ when $n=0$.

% \mytodo{For MLE, we use alternative criteria to measure the result of mu and sigma.}
\section{Micro experiments}
Before we move the evaluation the performance of HNN, we want to show that the initial states of HNN don't matter if we have enough large data sets. Also, we will show the fact that $G^{-1}$ can be approximate by $F$ within tolerant error.

\subsection{Initial states}
We use data set $P$ and set different initial states in predicting phase.
\mytableref{tbl:chapter5:initial-states} shows the influence of different initial states. We see incorrect initial states have bad impacts on RMSE. But these impacts cannot be ignored after 150 time steps. From \myfigref{fig:chapter5:initial-states}, we see that the predictive curves with different initial states are overlapping exactly after 150 time steps. 

In the following analysis, we always set the correct initial states for predicting phase.

\begin{table}[htb!]
\centering
\begin{adjustbox}{angle=0}
\begin{tabular}{||c|c|c|c|c||}
\hline 
    Data sets & Initial state & RMSE & RMSE(50) & RMSE(150) \\
\hline \hline
\multirow{4}{4em}{$P$} & 1 & 4.97 $\pm$ 0.04 &  2.52 $\pm$ 0.03 & 0.15 $\pm$ 0.03 \\
                           & 100 & 4.97 $\pm$ 0.04 & 2.52 $\pm$ 0.03 & 0.15 $\pm$ 0.03 \\
                           & -1 & 0.69 $\pm$ 0.04 & 0.51 $\pm$ 0.03 & 0.15 $\pm$ 0.03 \\
                           & -100 & 2.90 $\pm$ 0.03 & 1.20 $\pm$ 0.01 & 0.15 $\pm$ 0.03 \\
                           & \textbf{correct} & \textbf{0.15 $\pm$ 0.04} & \textbf{0.15 $\pm$ 0.03} & \textbf{0.15 $\pm$ 0.03} \\
\hline
\hline
\end{tabular}
\end{adjustbox}
\caption{Influence of initial state for predictions}
\label{tbl:chapter5:initial-states}
\end{table}


\begin{figure}[h!]
    \centering
    \includegraphics[height=5cm,width=\textwidth]{thesis/img/debug-initial-state-pavel-seq.pdf}
    \caption{Different initial states for set $P$}
    \label{fig:chapter5:initial-states}
\end{figure}

\subsection{Inverse of HNN}

% We call HNN as $G$ and the inverse function of $G$ is $F$.
% In this section, we show that the inverse $G$ can be approximated by another network $F$ (trained as \ref{xxx}). 
% is an experiment to see whether HNN is inverse or not.

% Before we move the evaluation the performance of HNN, we want to show that the initial states of HNN don't matter if we have enough large data sets. Also, we will show the fact that $G^{-1}$ can be approximate by $F$ within tolerant error.

% \textbf{Data sets.} In order to show the HNN is inverse, we generate two data sets by $G$ networks and switch the $outputs$ and $inputs$ of them, we call $\bar{P}$ and $\bar{S}$. Finally, we train HNN network $F$ using method shown in \mysectionref{sec:chapter2:training-pi-network}.
% $\bar{P}$ respectively.
\textbf{Setup.} One uses the data sets $S$ and $P$ generated from $G$ network and switch the $inputs$ and $outputs$ of them. We call these new data sets $\bar{S}$ and $\bar{P}$ respectively. One can train the $F$ network with $\bar{S}$ and $\bar{P}$ by training approach provided in \mysectionref{sec:chapter3:training-pi-network} and compare the $outputs$ of $F$ network with $inputs$ in $G$ network.
\newline
\textbf{Results.} \mytableref{tbl:chapter5:inverse-of-HNN} shows \mytodo{analysis here}.  
\myfigref{fig:chapter5:inverse-hnn-confidence-band} shows that all prediction results tightly bound 

\begin{table}[htb!]
\centering
\begin{adjustbox}{angle=0}
\begin{tabular}{||c|c|c|c||}
\hline 
Data sets & Length & hyper parameters & RMSE \\
\hline \hline
\multirow{3}{4em}{$\bar{S}$ set} & \multirow{1}{4em}{100} & (50, 50) & 0 $\pm$ 0 \\ 
                          \cline{2-4}
                          
                          & \multirow{1}{4em}{1000}  & (50, 50) & 0 $\pm$ 0 \\ 
                          \cline{2-4}
                          
                          & \multirow{1}{4em}{5000}  & (50, 50) & 0 $\pm$ 0 \\ 
                          \cline{2-4}
                          
% \hline                          
% \multirow{3}{4em}{$\bar{P}$ set} & \multirow{1}{4em}{100}  & (50, 50) & 0 $\pm$ 0 \\ 

%                           \cline{2-4}
                          
%                           & \multirow{1}{4em}{1000} & (50, 50) &  0.35 $\pm$ 0.06 \\ 
                      
%                           \cline{2-4}
                          
%                           & \multirow{1}{4em}{5000}  & (50, 50) & 0 $\pm$ 0 \\ 
%                           \cline{2-4}                          
\hline
\hline
\end{tabular}
\end{adjustbox}
\caption{RMSE for the data sets $\bar{S}$}
\label{tbl:chapter5:inverse-of-HNN}
\end{table}

\begin{figure}[htb!]
    \centering
    \subfloat[]{
    \includegraphics[width=\textwidth/3]{thesis/img/inverse-hnn-pavel-seq-__units__-25-__nb_plays__-25.pdf}
    }
    \subfloat[]{
    \includegraphics[width=\textwidth/3]{thesis/img/inverse-hnn-pavel-seq-__units__-25-__nb_plays__-25.pdf}
    }
    \subfloat[]{
    \includegraphics[width=\textwidth/3]{thesis/img/inverse-hnn-pavel-seq-__units__-25-__nb_plays__-25.pdf}
    }
    \caption{The results of confidence band. The curve in red ground-truth $inputs$ in $G$ network and the region in light blue is the maximal and minimal prediction of $F$ network at each time step.}
    \label{fig:chapter5:inverse-hnn-confidence-band}
\end{figure}
\section{Hysteretic loop}

We \myupdate{train} each network 20 times \myupdate{on train sets with different random seeds} and calculate the average and the standard deviation of metric \myupdate{on test sets} to \myupdate{see} if a network produces stable results.
\mytableref{tbl:chapter5:dima-pavel-seq-results} shows the metric of the different methods. We exploit hyper parameters and select the best results achieved by those methods.
We mark a result in bold if \myupdate{RMSE} is 
%\todo{I think if the RMSE if 5 times less than the other one, then it's significantly better.} 
significantly \myupdate{smaller} than other methods. We see that the HNN network achieves the best RMSE on all the data sets, which indicates it fits the hysteretic process best. Additionally, the amount of parameters HNN used is only about $1/9$ than that LSTM networks used. 

\myfigref{fig:chapter5:dima-seq-prediction-outputs-vs-time-steps} and \myfigref{fig:chapter5:pavel-seq-prediction-outputs-vs-time-steps} show the predictions on set $D$ and $P$ at each time step respectively. The larger blue areas indicate the higher uncertainty of the predictions, which means the performance of the network is worse. 

In \myfigref{fig:chapter5:dima-lstm-results}, we see the predictions of LSTM networks perform extremely poor considering the \myupdate{time steps from 601 to 900} corresponding to the micro-loops. The main reason is that LSTM networks do not take hysteretic properties into consideration. On the contrary, the blue areas restricted around the ground-truth curve tightly reveal that the HNN can reconstruct micro-loops from the training data set without observing them before. We can see that both LSTM networks and HNN predict well for time steps ranges from 900 to 950 since these data are presented in the training set. For the last 50 time steps, it corresponds to the macro-loops (see \myfigref{fig:chapter1:hysteresis-loop}) in the hysteretic process. Similarly, we see that HNN outperforms LSTM networks.

For the performance of set $P$, we see that the blue areas in \myfigref{fig:chapter5:pavel-lstm-results} are significantly larger than those in \myfigref{fig:chapter5:pavel-hnn-results}, especially for the first 100 time steps in test set. We see the noise is spread in the training set evenly (repeating with standard deviation $0.1, 0.5, 1, 2, 3, 4, 5$ every seven steps.) in set $P$. However, the first 100 data points are mainly dominated by $5 \cos(0.1n)$ due to the small standard deviation ($\sigma=0.1$), which is profoundly distinct from the training set. LSTM networks cannot deal with such a distribution in the test set well and predict results with high uncertainty. Whereas, HNN treats the training set as a hysteretic process and inspects the intrinsic properties of this dynamics successfully. That is why the blue areas of HNN are tightly adhered to with the ground-truth curve. 

\myfigref{fig:chapter5:dima-hnn-lstm-dynamics} and \myfigref{fig:chapter5:pavel-hnn-lstm-dynamics} show the \myupdate{trajectory} of set $D$ and $P$ individually. We apply cubic interpolation algorithm (see \myappendixsectionref{appendix:interpolation}) on $inputs$ of test set and generate new interpolated $inputs$. After that, we feed the interpolated $inputs$ into the trained networks to generate the interpolated predictive results. 

For data set $D$, we see the red curve (generated by LSTM networks) is oscillated left and right horizontally. But, the green curve  (generated by HNN) can follow the dynamics of the blue one in \myfigref{fig:chapter5:dima-hnn-lstm-dynamics-1,fig:chapter5:dima-hnn-lstm-dynamics-2,fig:chapter5:dima-hnn-lstm-dynamics-3,fig:chapter5:dima-hnn-lstm-dynamics-4,fig:chapter5:dima-hnn-lstm-dynamics-5}. We emphasize that the micro-loops in \myfigref{fig:chapter5:dima-hnn-lstm-dynamics-1,fig:chapter5:dima-hnn-lstm-dynamics-2,fig:chapter5:dima-hnn-lstm-dynamics-3,fig:chapter5:dima-hnn-lstm-dynamics-4,fig:chapter5:dima-hnn-lstm-dynamics-5} aren't exposed to the model during training. That's why the dynamic curve of LSTM networks is a horizontal line. Conversely, HNN reconstructs this kind of hysteretic process properly from the training set even though there still remains some bias. Particularly, \myfigref{fig:chapter5:dima-hnn-lstm-dynamics-9} is associated with the dynamics of the last 50 points in the test set, \myupdate{which corresponds to macro-loops}, we see that HNN is also able to reconstruct the macro-loops. 

As for data set $P$, we see the green curve (generated by HNN) and blue one (ground-truth curve) are overlapping precisely. At the same time, the red curve, which is generated by LSTM networks, deviates from the trace of the blue one. That confirms HNN does capture in micro-loops in a hysteretic process, but LSTM networks fail. 
Interestingly, \myfigref{fig:chapter5:pavel-hnn-lstm-dynamics-1} and \myfigref{fig:chapter5:pavel-hnn-lstm-dynamics-2} show that the red curve oscillates around the middle of the micro loop. It implies the fact that LSTM networks try to predict the average of a hysteretic process.

% \mytodo{to update, both predict based on the training set.}
% Overall, the HNN inspects the hysteretic data set and reconstructs this process correctly. Whereas, LSTM networks predict the results based on what it has seen in the training set (see \myfigref{fig:chapter5:dima-hnn-lstm-dynamics-8,fig:chapter5:dima-lstm-results}) and fail to rebuild the micro-loops and macro-loops.
\myupdate{Overall, the HNN inpsects the hysteretic data set and reconstructs macro-loops and micro-loops correctly. Whereas, LSTM networks fails to reconstruct this process due to lack of the assumptions on hysteretic data sets. Moreover, HNN fits the data sets better with fewer parameters compared with LSTM/SimpleRNN/GRU.}

\begin{table}[htb!]
\centering
\begin{adjustbox}{angle=0}
\begin{tabular}{||c|c|cc||}
\hline 
Data sets & networks & \#parameters & RMSE \\
\hline \hline
\multirow{2}{4em}{$D$ set} 
                             & \multirow{1}{6em}{SimpleRNN} & 16769 & 12.62 $\pm$ 1.50  \\             
\cline{2-4}

                             & \multirow{1}{6em}{GRU} & 50049 & 12.13 $\pm$ 2.64 \\ 
\cline{2-4}
                             & \multirow{1}{6em}{LSTM} & 66560 & 14.39 $\pm$ 4.07 \\
\cline{2-4}
                             & \multirow{1}{6em}{HNN} & 7501 & \textbf{2.79 $\pm$ 0.09} \\ 
\hline \hline
\multirow{2}{4em}{$P$ set} 

% 
                            & \multirow{1}{6em}{SimpleRNN} & 16769 & 5.75 $\pm$ 0.66 \\
\cline{2-4}

                             & \multirow{1}{6em}{GRU} & 50049 & 4.70 $\pm$ 0.45 \\ 
\cline{2-4}
                            & \multirow{1}{6em}{LSTM}   & 66560 & 4.99 $\pm$ 0.33 \\ 

\cline{2-4}
                         & \multirow{1}{6em}{HNN} 
                         & 7501 & \textbf{0.09 $\pm$ 0.02} \\ 
\hline
\end{tabular}
\end{adjustbox}
\caption[The RMSE for the test sets $D$ and $P$]{The RMSE for the test sets $D$ and $P$. The total number of parameters used to generate ground-truth is 7501 (50 $nb\_plays$ and 50 $units$).}
\label{tbl:chapter5:dima-pavel-seq-results}
\end{table}

\begin{figure}[h!]
    \centering
    \subfloat[LSTM]{
        \includegraphics[width=\textwidth/2]{thesis/img/debug-lstm-dima-__units__-128.pdf}
        \label{fig:chapter5:dima-lstm-results}
    }
    \subfloat[HNN]{
        \includegraphics[width=\textwidth/2]{thesis/img/debug-hnn-dima-__units__-25-__nb_plays__-25.pdf}
        \label{fig:chapter5:dima-hnn-results}
    }
    \caption[Predictions for test set $D$.]{Predictions for test set $D$. The curve in red is the ground-truth test set and the light blue areas correspond to the uncertainty of the predictive results. \myfigref{fig:chapter5:dima-lstm-results} is the predictions of LSTM networks and \myfigref{fig:chapter5:dima-hnn-results} is the predictions of HNN.}
    \label{fig:chapter5:dima-seq-prediction-outputs-vs-time-steps}
\end{figure}

\begin{figure}[h!]
    \centering
    \subfloat[LSTM]{
        \includegraphics[width=\textwidth]{thesis/img/debug-lstm-pavel-__units__-128.pdf}
        \label{fig:chapter5:pavel-lstm-results}
    }
    \hfill
    \subfloat[HNN]{
        \includegraphics[width=\textwidth]{thesis/img/debug-hnn-pavel-__units__-25-__nb_plays__-25.pdf}
        \label{fig:chapter5:pavel-hnn-results}
    }
    \caption[Predictions for test set $P$.]{Predictions for test set $P$. The curve in red is the ground-truth test set and the light blue areas correspond to the uncertainty of the predictive results. \myfigref{fig:chapter5:pavel-lstm-results} is the predictions of LSTM networks and \myfigref{fig:chapter5:pavel-hnn-results} is the predictions of HNN.}
    \label{fig:chapter5:pavel-seq-prediction-outputs-vs-time-steps}
\end{figure}


\begin{figure}[h]
    \subfloat[]{
    \includegraphics[width=\textwidth/3]{debug-dima-hnn-lstm-inspection/1.pdf}
    \label{fig:chapter5:dima-hnn-lstm-dynamics-1}

    }
    \subfloat[]{
    \includegraphics[width=\textwidth/3]{debug-dima-hnn-lstm-inspection/2.pdf}
    \label{fig:chapter5:dima-hnn-lstm-dynamics-2}
    }
    \subfloat[]{
    \includegraphics[width=\textwidth/3]{debug-dima-hnn-lstm-inspection/3.pdf}
    \label{fig:chapter5:dima-hnn-lstm-dynamics-3}
    
    }
    \hfill
    \subfloat[]{
    \includegraphics[width=\textwidth/3]{debug-dima-hnn-lstm-inspection/4.pdf}
    \label{fig:chapter5:dima-hnn-lstm-dynamics-4}

    }
    \subfloat[]{
    \includegraphics[width=\textwidth/3]{debug-dima-hnn-lstm-inspection/5.pdf}
    \label{fig:chapter5:dima-hnn-lstm-dynamics-5}

    }
    \subfloat[]{
    \includegraphics[width=\textwidth/3]{debug-dima-hnn-lstm-inspection/6.pdf}
    \label{fig:chapter5:dima-hnn-lstm-dynamics-6}
    
    }
    \hfill
    \subfloat[]{
    \includegraphics[width=\textwidth/3]{debug-dima-hnn-lstm-inspection/7.pdf}
    \label{fig:chapter5:dima-hnn-lstm-dynamics-7}
    
    }
    \subfloat[]{
    \includegraphics[width=\textwidth/3]{debug-dima-hnn-lstm-inspection/8.pdf}
    \label{fig:chapter5:dima-hnn-lstm-dynamics-8}

    }
    \subfloat[]{
    \includegraphics[width=\textwidth/3]{debug-dima-hnn-lstm-inspection/9.pdf}
    \label{fig:chapter5:dima-hnn-lstm-dynamics-9}
    
    }
    \caption[The animation of hysteretic loops for set $D$. ]{The animation of hysteretic loops for set $D$. \myupdate{Different plots correspond to different time steps. The last time step of the (blue/green/red) curves in each plot is monotonically increasing. The scatter points in cyan consist of ground-truth and predicted test sets.} The curve in blue is the animation of ground-truth data sets and the one in red is the animation generated by LSTM networks and the one in green is the animation generated by HNN.}
    \label{fig:chapter5:dima-hnn-lstm-dynamics}
\end{figure}


%% lstm & hnn motion
\begin{figure}[h]
    \subfloat[]{
    \includegraphics[width=\textwidth/3]{debug-pavel-hnn-lstm-inspection/1.pdf}
    \label{fig:chapter5:pavel-hnn-lstm-dynamics-1}

    }
    \subfloat[]{
    \includegraphics[width=\textwidth/3]{debug-pavel-hnn-lstm-inspection/2.pdf}
    \label{fig:chapter5:pavel-hnn-lstm-dynamics-2}
    }
    \subfloat[]{
    \includegraphics[width=\textwidth/3]{debug-pavel-hnn-lstm-inspection/3.pdf}
    \label{fig:chapter5:pavel-hnn-lstm-dynamics-3}
    
    }
    \hfill
    \subfloat[]{
    \includegraphics[width=\textwidth/3]{debug-pavel-hnn-lstm-inspection/4.pdf}
    \label{fig:chapter5:pavel-hnn-lstm-dynamics-4}

    }
    \subfloat[]{
    \includegraphics[width=\textwidth/3]{debug-pavel-hnn-lstm-inspection/5.pdf}
    \label{fig:chapter5:pavel-hnn-lstm-dynamics-5}

    }
    \subfloat[]{
    \includegraphics[width=\textwidth/3]{debug-pavel-hnn-lstm-inspection/6.pdf}
    \label{fig:chapter5:pavel-hnn-lstm-dynamics-6}
    
    }
    \hfill
    \subfloat[]{
    \includegraphics[width=\textwidth/3]{debug-pavel-hnn-lstm-inspection/7.pdf}
    \label{fig:chapter5:pavel-hnn-lstm-dynamics-7}
    
    }
    \subfloat[]{
    \includegraphics[width=\textwidth/3]{debug-pavel-hnn-lstm-inspection/8.pdf}
    \label{fig:chapter5:pavel-hnn-lstm-dynamics-8}

    }
    \subfloat[]{
    \includegraphics[width=\textwidth/3]{debug-pavel-hnn-lstm-inspection/9.pdf}
    \label{fig:chapter5:pavel-hnn-lstm-dynamics-9}
    
    }
    \caption[The animation of hysteretic loops for set $P$.]{The animation of hysteretic loops for set $P$. \myupdate{Different plots correspond to different time steps. The last time step of the (blue/green/red) curves in each plot is monotonically increasing. The scatter points in cyan consist of ground-truth and predicted test sets.} The curve in blue is the animation of ground-truth data sets and the one in red is the animation generated by LSTM networks and the one in green is the animation generated by HNN.}
    \label{fig:chapter5:pavel-hnn-lstm-dynamics}
\end{figure}

\FloatBarrier
% \newpage
% \section{MLE}

\textbf{Data set.} outputs is a random walk and input is generated from inverse network $F$

% \mytodo{same weights but different nb plays}
% \mytodo{different weights and different plays  no noise}
% \mytodo{different weights  and different plays noise} 
% \section{hnn-lstm-all-sigma-0}

\begin{table}[h]
\begin{adjustbox}{angle=0}
\begin{tabular}{||c|c|c|c||}
\hline 
\#play (\#unit) & network & \#parameters & RMSE 
\\ \hline
%%%%%%%%%%%%%%%%%%%%%%%%%%%%%%%%%%%%%%%%%%%%%%
1& lstm-unit-1 & 12 & 0.293196103
\\ \hline
1 & lstm-unit-8 & 320 & 0.3420867
\\ \hline
1 & lstm-unit-16 & 1,152 & 0.288819476
\\ \hline
1 & lstm-unit-32 & 4,352 & 0.291840419
\\ \hline
1 & lstm-unit-64 & 16,896 & 0.453519368
\\ \hline
1 & lstm-unit-128 & 66,560 &  0.261012763
\\ \hline
1 & lstm-unit-256 & 264,192 & 0.387604944
\\ \hline
%%%%%%%%%%%%%%%%%%%%%%%%%%%%%%%%%%%%%%%%%%%%%%
50 & lstm-unit-1 & 12 & 7.24246318
\\ \hline
50 & lstm-unit-8 & 320 & 11.47440343
\\ \hline
50 & lstm-unit-16 & 1,152 & 6.699965269
\\ \hline
50 & lstm-unit-32 & 4,352 & 5.361344497
\\ \hline
50 & lstm-unit-64 & 16,896 & 6.257610363
\\ \hline
50 & lstm-unit-128 & 66,560 & 16.18529624
\\ \hline
50 & lstm-unit-256 & 264,192 & 14.22657381
\\ \hline
%%%%%%%%%%%%%%%%%%%%%%%%%%%%%%%%%%%%%%%%%%%%%%
100 & lstm-unit-1 & 12 & 23.56603552
\\ \hline
100 & lstm-unit-8 & 320 & 7.714270118
\\ \hline
100 & lstm-unit-16 & 1,152 & 9.293488583
\\ \hline
100 & lstm-unit-32 & 4,352 & 8.239636524
\\ \hline
100 & lstm-unit-64 & 16,896 & 42.83023104
\\ \hline
100 & lstm-unit-128 & 66,560 &  40.50239738
\\ \hline
100 & lstm-unit-256 & 264,192 & 31.05811984
\\ \hline

%%%%%%%%%%%%%%%%%%%%%%%%%%%%%%%%%%%%%%%%%%%%%%
% 500 & lstm-unit-1 & -1 & 14.27020212
% \\ \hline
% 500 & lstm-unit-8 & -1 & 15.04781055
% \\ \hline
% 500 & lstm-unit-16 & -1 & 18.72035688
% \\ \hline
% 500 & lstm-unit-32 & -1 & 13.1013827
% \\ \hline
% 500 & lstm-unit-64 & -1 & 42.8302338
% \\ \hline
% 500 & lstm-unit-128 & -1 &  40.50240013
% \\ \hline
% 500 & lstm-unit-256 & -1 & 40.50240013
% \\ \hline

\end{tabular}
\end{adjustbox}
\caption{lstm same weights no noise}
\end{table}

% currently, it is given by the following formula
% \begin{equation}
%         \#parameter = \#play * (3\#unit+1)
% \end{equation}

\begin{table}[h]
\begin{adjustbox}{angle=0}
\begin{tabular}{||c|c|c|c||}
\hline 
\#play (\#unit) & network & \#parameters & RMSE 
\\ \hline
%%%%%%%%%%%%%%%%%%%%%%%%%%%%%%%%%%%%%%%%%%%%%%
1& hnn-play-1-unit-1 & 4 & 0.181846233
\\ \hline
1 & hnn-play-2-unit-2 & 13 & 0.18374151
\\ \hline
1 & hnn-play-8-unit-8 & 193 & 0.168357833
\\ \hline
%%%%%%%%%%%%%%%%%%%%%%%%%%%%%%%%%%%%%%%%%%%%%%

50 & hnn-play-50-unit-50 & 7,501 & 6.352061288
\\ \hline
50 & hnn-play-100-unit-100 & 30,001 & 6.18795001

\\ \hline
50 & hnn-play-200-unit-200 & 120,001 &  6.163790714

\\ \hline
%%%%%%%%%%%%%%%%%%%%%%%%%%%%%%%%%%%%%%%%%%%%%%

100 &  hnn-play-50-unit-50 & 7,501 & 12.71635473
\\ \hline
100 &  hnn-play-100-unit-100 & 30,001 & 12.42587169

\\ \hline
100 &  hnn-play-200-unit-200 & 120,001 & 11.97417496
\\ \hline
\end{tabular}
\end{adjustbox}
\caption{hnn same weights no noise using elu as activation}
\end{table}
% \section{different-weights-no-noise, sigma-0}
\begin{table}
\begin{adjustbox}{angle=0}
\begin{tabular}{||c|c|c|c||}
\hline 
\#play (\#unit) & network & \#parameters & RMSE 
\\ \hline
%%%%%%%%%%%%%%%%%%%%%%%%%%%%%%%%%%%%%%%%%%%%%%
50 & lstm-unit-1 & 12 & 2.210005733
\\ \hline
50 & lstm-unit-8 & 320 & 2.96759788

\\ \hline
50 & lstm-unit-16 & 1,152 & 2.138047736

\\ \hline
50 & lstm-unit-32 & 4,352 & 2.385471522

\\ \hline
50 & lstm-unit-64 & 16,896 & 2.60343689

\\ \hline
50 & lstm-unit-128 & 66,560 &  2.176426255

\\ \hline
50 & lstm-unit-256 & 264,192 & 2.176366421

\\ \hline
%%%%%%%%%%%%%%%%%%%%%%%%%%%%%%%%%%%%%%%%%%%%%%
100 & lstm-unit-1 & 12 & 4.792876647

\\ \hline
100 & lstm-unit-8 & 320 & 3.100921422

\\ \hline
100 & lstm-unit-16 & 1,152 & 1.459150006

\\ \hline
100 & lstm-unit-32 & 4,352 & 2.49986648

\\ \hline
100 & lstm-unit-64 & 16,896 & 2.514736698

\\ \hline
100 & lstm-unit-128 & 66,560 & 2.105919619

\\ \hline
100 & lstm-unit-256 & 264,192 & 2.105532431
\\ \hline
\end{tabular}
\end{adjustbox}
\caption{lstm different weights no noise}
\end{table}


\begin{table}
\begin{adjustbox}{angle=0}
\begin{tabular}{||c|c|c|c||}
\hline 
\#play (\#unit) & network & \#parameters & RMSE 
\\ \hline
% %%%%%%%%%%%%%%%%%%%%%%%%%%%%%%%%%%%%%%%%%%%%%%
% 1& hnn-play-1-unit-1 & 4 & 0.181846233
% \\ \hline
% 1 & hnn-play-2-unit-2 & 13 & 0.18374151
% \\ \hline
% 1 & hnn-play-8-unit-8 & 193 & 0.168357833
% \\ \hline
%%%%%%%%%%%%%%%%%%%%%%%%%%%%%%%%%%%%%%%%%%%%%%
50 & hnn-play-10-unit-10 & 301 & 0.520737239
\\ \hline
50 & hnn-play-25-unit-10 & 751 & 0.481243223

\\ \hline
50 & hnn-play-25-unit-25 & 1,876 & 0.657608419

\\ \hline
50 & hnn-play-50-unit-50 & 7,501 &  0.678940872

\\ \hline
50 & hnn-play-100-unit-50 & 15,001 &  0.669122054


\\ \hline
%%%%%%%%%%%%%%%%%%%%%%%%%%%%%%%%%%%%%%%%%%%%%%

100 &  hnn-play-25-unit-25 & 1,876 & 0.650535543

\\ \hline
100 &  hnn-play-50-unit-25 & 3,751 & 0.618666728

\\ \hline
100 &  hnn-play-50-unit-50 & 7,501 & 0.642434578

\\ \hline
100 &  hnn-play-100-unit-50 & 15,001 & 0.65746967

\\ \hline
100 &  hnn-play-100-unit-100 & 30,001 & 0.639300454

\\ \hline
100 &  hnn-play-200-unit-25 & 15,001 & 0.343120708

\\ \hline
100 &  hnn-play-200-unit-100 & 60,001 & 0.466267496

\\ \hline
\end{tabular}
\end{adjustbox}
\caption{hnn different weights no noise using elu as activation}
\end{table}

% \section{different-weights-with-noise, sigma-8}
% \begin{table}
% \begin{adjustbox}{angle=0}
% \begin{tabular}{||c|c|c|c||}
% \hline 
% \#play (\#unit) & network & \#parameters & RMSE 
% \\ \hline
% %%%%%%%%%%%%%%%%%%%%%%%%%%%%%%%%%%%%%%%%%%%%%%
% 50 & lstm-unit-1 & 12 & 3.444246389
% \\ \hline
% 50 & lstm-unit-8 & 320 & 1.463447021
% \\ \hline
% 50 & lstm-unit-16 & 1,152 & 1.328417165
% \\ \hline
% 50 & lstm-unit-32 & 4,352 & 1.303146381
% \\ \hline
% 50 & lstm-unit-64 & 16,896 & 1.201215367
% \\ \hline
% 50 & lstm-unit-128 & 66,560 &  1.398202932
% \\ \hline
% 50 & lstm-unit-256 & 264,192 & 1.257714804
% \\ \hline
% %%%%%%%%%%%%%%%%%%%%%%%%%%%%%%%%%%%%%%%%%%%%%%
% 100 & lstm-unit-1 & 12 & 6.104827801
% \\ \hline
% 100 & lstm-unit-8 & 320 & 2.496286256

% \\ \hline
% 100 & lstm-unit-16 & 1,152 & 2.613680848

% \\ \hline
% 100 & lstm-unit-32 & 4,352 & 2.75013601

% \\ \hline
% 100 & lstm-unit-64 & 16,896 & 2.73756359

% \\ \hline
% 100 & lstm-unit-128 & 66,560 & 2.574973189

% \\ \hline
% 100 & lstm-unit-256 & 264,192 & 2.56076419

% \end{tabular}
% \end{adjustbox}
% \caption{lstm different weights  noise}
% \end{table}


% \begin{table}
% \begin{adjustbox}{angle=0}
% \begin{tabular}{||c|c|c|c||}
% \hline 
% \#play (\#unit) & network & \#parameters & RMSE 
% \\ \hline

% 50 & hnn-play-10-unit-10 & 301 & 0.600649707
% \\ \hline
% 50 & hnn-play-25-unit-10 & 751 & 0.590401014
% \\ \hline
% 50 & hnn-play-25-unit-25 & 1,876 & 0.640757911
% \\ \hline
% 50 & hnn-play-50-unit-50 & 7,501 &  0.570615017
% \\ \hline
% 50 & hnn-play-100-unit-50 & 15,001 &  0.582596486

% \\ \hline
% %%%%%%%%%%%%%%%%%%%%%%%%%%%%%%%%%%%%%%%%%%%%%%
% 100 &  hnn-play-25-unit-25 & 1,876 & 0.801752993
% \\ \hline
% 100 &  hnn-play-50-unit-25 & 3,751 & 1.42465427
% \\ \hline
% 100 &  hnn-play-50-unit-50 & 7,501 & 0.959867101
% \\ \hline
% 100 &  hnn-play-100-unit-50 & 15,001 & 1.985093566
% \\ \hline
% 100 &  hnn-play-100-unit-100 & 30,001 & 0.436862487
% \\ \hline
% 100 &  hnn-play-200-unit-25 & 15,001 & 1.135538106
% \\ \hline
% 100 &  hnn-play-200-unit-100 & 60,001 & 0.844494969
% \\ \hline
% \end{tabular}
% \end{adjustbox}
% \caption{hnn different weights noise using elu as activation}
% \end{table}



\begin{table}[h]
\centering
\begin{adjustbox}{angle=0}
\begin{tabular}{||c|c|ccc||}
\hline 
\#plays & networks & hyper parameters & \#parameters & RMSE ($\mu \pm \sigma$)\\
\hline \hline
\multirow{13}{4em}{50} & \multirow{7}{4em}{LSTM}   & 1    & 12         & 3.444246389 \\ 
                         &                         & 8    & 320        & 1.463447021 \\ 
                         &                         & 16   & 1,152      & 1.328417165 \\ 
                         &                         & 32   & 4,352      & 1.303146381 \\ 
                         &                         & 64   & 16,896     & 1.201215367 \\ 
                         &                         & 128  & 66,560     & 1.398202932 \\ 
                         &                         & 256  & 264,192    & 1.257714804 \\ 
\cline{2-5}
                         & \multirow{6}{4em}{HNN}  & (10, 10)    & 301      & 0.600649707 \\
                         &                         & (25, 10)    & 751      & 0.590401014 \\ 
                         &                         & (25, 25)    & 1,876    & 0.640757911 \\ 
                         &                         & (50, 25)    & 7,501    & 0.570615017 \\ 
                         &                         & (100, 50)   & 15,001   & 0.582596486 \\ 
                         
\hline \hline
\multirow{13}{4em}{100} & \multirow{7}{4em}{LSTM}  & 1    & 12          & 6.104827801 \\ 
                         &                         & 8    & 320         & 2.496286256 \\ 
                         &                         & 16   & 1152        & 2.613680848 \\ 
                         &                         & 32   & 4352        & 2.75013601  \\ 
                         &                         & 64   & 16896       & 2.73756359  \\ 
                         &                         & 128  & 66560       & 2.574973189 \\ 
                         &                         & 256  & 264192      & 2.56076419  \\ 
\cline{2-5}
                         & \multirow{6}{4em}{HNN}  & (25, 25)     & 1,876    & 0.801752993 \\
                         &                         & (50, 50)     & 7,501    & 0.959867101 \\ 
                         &                         & (100, 50)    & 15,001   & 1.985093566 \\ 
                         &                         & (100, 100)   & 30,001   & 0.436862487 \\        
                         &                         & (200, 25)    & 15,001   & 1.135538106 \\        
                         &                         & (200, 100)   & 60,001   & 0.844494969 \\        

\hline
\end{tabular}
\end{adjustbox}
\caption{RMSE for the data sets D and P}
\end{table}
\newpage
\section{HNN for market model}\label{sec:chapter5:hnn-for-market-model}
In this section, we only compare HNN with LSTM network since SimpleRNN and GRU don't perform much better than LSTM.
\newline
\section{Synthetic data sets from financial market model}\label{sec:chapter5:synthetic-data-sets}
% \textbf{Synthetic data sets.}\label{sec:chapter5:synthetic-data-sets}
Following the practical approach described in     \mysectionref{sec:chapter3:practical-approches}, we obtain the following synthetic data sets (see \myfigref{fig:chapter5:market-ground-truth-dataset}). We only display the figures whose standard deviation of $b_n - b_{n-1}$ is 20 since the standard deviation 10 is similar to it. \myupdate{We use the first 1300 points for training set and the rest 400 points for test set.}
\begin{figure}[h!]
    \centering
    \subfloat[]{
        \includegraphics[height=7cm,width=\textwidth]{market-ground-truth-dataset}
    }
    \caption{The synthetic data sets generated from the financial market model. The standard deviation of difference of random walk $b_n - b_{n-1}$ is 20. The top one is the random walk, which is underline in the real market, of the simulation. The middle one is the fluctuation of price based on the model. And the bottom one the change of the total number of stocks in the market, which is following to random walk. }
    \label{fig:chapter5:market-ground-truth-dataset}
\end{figure}
\newline
% \textbf{Setup.} 
\section{Setup}
In the financial market model, we evaluate the LSTM networks with units 64, 128, and 256, and select the best performance. In this thesis, LSTM networks with 64 units \mydelete{beat}\myupdate{outperform} the rest models. As for HNN, we utilize 50 plays and 50 units as hyperparameters to train the network. All these networks are trained with \mydelete{enough}\myupdate{10000} epochs\mydelete{(10k epochs in this thesis)}. For RMSE, we evaluate $d_n = |b_n - b_{n-1}|$ instead of $b_n$ directly due to the loss function given in \mysectionref{sec:chapter4:direct_learning}. To make it clear, we rewrite RMSE as follows,
\begin{equation}
    \text{RMSE} = \sqrt{\frac{\sum_{i=1}^{N}(\hat{d}_i - d_i)^2}{N}}
    % \quad n=0,1,\ldots,N-1
\end{equation}
\newline\newline
% \textbf{Results.} 
\section{Results and analysis}
\mytableref{tbl:chapter5:simulation-stock-results} presents the different performance between LSTM networks and HNN. The data set with $\sigma=20$ means $d_n$ is followed to normal distribution $\mathcal{N}(0, 20)$. We see that HNN outperforms LSTM networks dramatically. HNN can recover the mean $\mu$ and standard deviation $\sigma$ from the random walk accurately, and the results are much more \mydelete{stable}\myupdate{robust} than that learned by LSTM networks \myupdate{comparing the standard deviation of RMSE}. Meanwhile, it reveals that HNN utilizes fewer parameters to produce better RMSE than LSTM.
\myfigref{fig:chapter5:lstm-hnn-stock-mle} shows the curve of $d_n$ generated by LSTM networks and HNN.
We notice that the average of standard deviation obtained by LSTM networks is around 15.24, which is underrated, and \mytableref{tbl:chapter5:simulation-stock-results} confirms this circumstance. Unlike the predicted curve produced by HNN overlapping with the ground-truth one, the blue curve generated by LSTM networks diverges from the ground-truth results and has a smaller magnitude spreading most timesteps. 
Finally \todo{Isn't it for given $(x_n, y_n)$ ? No. It's trained by MLE. The scale of outputs is  not the same} \myfigref{fig:chapter1:market-lstm-result,fig:chapter1:market-hnn-result} present that HNN traces the ground-truth random walk better than LSTM networks do.

\begin{table}[htb!]
\centering
\begin{adjustbox}{angle=0}
\begin{tabular}{||c|c|c|c|c|c||}
\hline 
 Data set & network & \#parameters & estimated $\mu$ & estimated $\sigma$ & RMSE  \\
\hline \hline
\multirow{2}{4em}{$\sigma=10$} & \multirow{1}{4em}{LSTM} & 16961 & 0.05 $\pm$ 0.20 & 5.55 $\pm$ 1.98 & 10.05 $\pm$ 1.39 \\ 
                          \cline{2-6}
                          & \multirow{1}{4em}{HNN}  & 7501 & -0.25  $\pm$ 0.44 & 12.22 $\pm$ 0.38 & 7.40 $\pm$ 0.84 \\ 
                          \cline{2-6}
\hline

\multirow{2}{4em}{$\sigma=20$} & \multirow{1}{4em}{LSTM} & 16961 & 0.05 $\pm$ 0.20 & 15.24 $\pm$ $4.26$ & 18.36 $\pm$ 2.05 \\
                          \cline{2-6}
                          & \multirow{1}{4em}{HNN}  & 7501 & 0.09 $\pm$ 0.20 & 19.74 $\pm$ 0.42  & 8.82 $\pm$ 0.91 \\ 
                          \cline{2-6}
\hline

% \multirow{2}{4em}{$\sigma=110$} & \multirow{1}{4em}{LSTM} & 0 & x & xxxx $\pm$ xxx (not precise)\\ 
%                           \cline{2-5}
%                           & \multirow{1}{4em}{HNN}  & 0 & x & 0 $\pm$ 0 \\ 
%                           \cline{2-5}
% \hline
\end{tabular}
\end{adjustbox}
\caption{RMSE for simulation results on test set} 
\label{tbl:chapter5:simulation-stock-results}
\end{table}
% \FloatBarrier
% \begin{figure}
%     \centering
%     \subfloat[]{
%         \includegraphics[width=\textwidth]{thesis/img/debug-lstm-stock-mle.pdf}
%     }
%     \caption{Caption}
%     \label{fig:chapter5:lstm-stock-mle}
% \end{figure}

\begin{figure}[htb!]
    \centering
    \subfloat[LSTM]{
        \includegraphics[height=3.5cm,width=\textwidth]{thesis/img/lstm-stock-diff-outputs-units-64.pdf}
        \label{fig:chapter5:lstm-stock-mle}
    }
    \hfill
    \subfloat[HNN]{
        \includegraphics[height=3.5cm,width=\textwidth]{thesis/img/hnn-stock-diff-outputs.pdf}
        \label{fig:chapter5:hnn-stock-mle}
    }    
    \caption{Sequence of difference of $b_{n} - b_{n-1}$ on test set. The curve in red is ground-truth and the blue one is generated by trained networks.}
    \label{fig:chapter5:lstm-hnn-stock-mle}
\end{figure} 


% \subsection{analysis-unknown-mu}
% \begin{table}[htb!]
% \centering
% \begin{adjustbox}{angle=0}
% \begin{tabular}{||c|c|c|c|c||}
% \hline 
% network & hyper parameters & \#parameters & RMSE  & predict mu \\
% \hline \hline
% % \multirow{3}{4em}{$\bar{S}$ set} & \multirow{1}{4em}{100} & (25,25) & 0 $\pm$ 0 \\ 
% %                           \cline{2-4}
% %                           & \multirow{1}{4em}{1000}  & (25,25) & 0 $\pm$ 0 \\ 
% %                           \cline{2-4}
% %                           & \multirow{1}{4em}{10000}  & (25,25) & 0 $\pm$ 0 \\ 
% %                           \cline{2-4}
% \hline
% \hline
% \end{tabular}
% \end{adjustbox}
% \caption{RMSE for simulation results}
% \label{tbl:chapter5:simulation-stock-results}
% \end{table}

% \mytodo{add mle loss curve to see that mle in hnn is futhur smaller than lstm}
% \mytodo{add reconstructed agents behaviours graphs}

% \mytodo{show that hnn can reconstruct agents behavior}

\FloatBarrier
% \textbf{Reconstructing \myupdate{aggregated} dynamics of agents.} 
\section{Reconstructing \myupdate{aggregated} dynamics of agents}
\myfigref{fig:chapter5:dynamics-of-agents} displays the dynamics of agents retrieved from the financial market model and HNN. 
The red dotted curve is the dynamics extracted from HNN, and one in orange and blue is the dynamics from the market model. The blue curve is the dynamics of agents taken place if external agents buy (sell) stocks in simulation. Furthermore, the orange one is that we want to inspect what the dynamics if the external agents sell (buy) stocks at the same time steps. \myupdate{The intersection of blue and orange curve is the initial price at that timestep} We stress that the data in orange doesn't show in the training set. \myfigref{fig:chapter5:dynamics-of-agents-1} indicates that HNN can reconstruct the dynamics of agents thoroughly from the training set. 
\myfigref{fig:chapter5:dynamics-of-agents-2} shows the volume of agents decreases slightly, then increases. Similarly, the numeric quantity generated from HNN fits the trajectory. 
The orange curve in \myfigref{fig:chapter5:dynamics-of-agents-3} is unconscious to the training set. However, we observe the red curve produced by HNN reconstructs these dynamics correctly. It implies HNN not only inspects the training set but also restores the agents participated in the market.
Considering \myfigref{fig:chapter5:dynamics-of-agents-4}, we observe the blue curve decreases first, then increases, finally decreases. Meantime the trajectory obtained from HNN traces this tendency. 
Moreover, we can explain that the avalanche of the price based on \myfigref{fig:chapter5:dynamics-of-agents-3,fig:chapter5:dynamics-of-agents-4}. We see there are black and red horizontal lines in \myfigref{fig:chapter5:dynamics-of-agents-3}. If the random walk decreases from -457 to -570 (the black line), then the price only increases from 0.008 to 0.056. However, if the next random walk is a bit lower than -570, take -590 (the red line) for example, we will observe that the price will rise to 0.165, which is triple times than the previous case. That illustrates why some small fluctuations of stocks lead to a large changes in price. 

% Suppose that the change of stocks in market

% The difference between blue and orange curve is that the blue curve is the dynamics of external agents 

\begin{figure}[htb!]
    \centering
    \subfloat[]{
    \includegraphics[width=\textwidth/2]{thesis/img/inspect-agents-behaviours/29.pdf}
        \label{fig:chapter5:dynamics-of-agents-1}

    }
    \subfloat[]{
    \includegraphics[width=\textwidth/2]{thesis/img/inspect-agents-behaviours/31.pdf}
            \label{fig:chapter5:dynamics-of-agents-2}
    }
    \hfill
    \subfloat[]{
    \includegraphics[width=\textwidth/2]{thesis/img/inspect-agents-behaviours/15.pdf}
    \label{fig:chapter5:dynamics-of-agents-3}
    }
    \subfloat[]{
    \includegraphics[width=\textwidth/2]{thesis/img/inspect-agents-behaviours/53.pdf}
    \label{fig:chapter5:dynamics-of-agents-4}
    }    
    \caption{Dynamics of aggregated agents. If the amount of stocks increases, it means the external agents sell stocks to the market, and the price drops. On the contrary, if the amount of stocks decreases, it indicates the external agents buy stocks from the market, and the price rises. 
    \mydelete{The numeric value of the number of agents doesn't matter but the difference between two consecutive time steps, $b_{n} - b_{n-1}$, matters.} \myupdate{The blue curve (dynamics generated by market model I) is the dynamics of agents that took place in simulation. The orange one (dynamics generated by market model II) is that we want to inspect what the dynamics would be if external agents take the opposite action at the same time steps, i.e., if the blue curve is resulted in by external agents buy the stocks from market, the orange curve is resulted in by the external agents sell the stock to market.}}
    \label{fig:chapter5:dynamics-of-agents}
\end{figure}
\FloatBarrier

% \subsection{Prediction}  
% In this evaluation, we only try to predict the trend of next price, up or down, in the next time step. And we also use the following LSTM network architecture to predict the next price.

% \textbf{LSTM network architecture.} 


% In order the compare the results of LSTM
% \mytodo{show that hnn can give a distribution of price, not average of price} 

\section{Predicting price}
\textbf{Methods.} To distinguish difference predicting performance among different methods, we consider the following three approaches. The first approach is to predict the next price $\hat{p}_i$ as the previous price $p_{i-1}$, and we call it \textit{native method}. The second method is using LSTM networks described in \citep{kagglelstm}. The last one is using the methods described in \mysectionref{sec:chapter4:predicting_price}.  
 
\textbf{Measure.} To show the fine performance of the predictions, we divide \myupdate{price change} ${p_n-p_{n-1}}$ and \myupdate{noise jumps $b_n - b_{n-1}$} into $R \times R$ grids 
\mydelete{and generate the difference $|\hat{p}_i - p_i|$ at each}
slot $\text{SLOT}_j$ \myupdate{where}, 


\begin{equation}
\begin{aligned}
\text{SLOT}_j = \big\{ (\Delta p, \Delta b) : \Delta p_{j-1} < \Delta p \le \Delta p_j, \Delta b_{j-1} < \Delta b \le \Delta b_j \big\}
\end{aligned}
\end{equation}
and $\hat{p}_i$ is the predicted price, $p_i$ is the ground-truth price, $\Delta{p}_j = j \frac{\max\{p_1, p_2, \ldots\} - \min\{p_1, p_2, \ldots\}}{R}$, $\Delta{b}_j = j \frac{\max\{b_1, b_2, \ldots\} - \min\{b_1, b_2, \ldots\}}{R}$ and $j=0,1,\ldots$

Then the modified RMSE at 
$\text{SLOT}_j$ is given by,
\begin{equation}
\begin{aligned}
& & & \text{RMSE}(\text{SLOT}_j) = \sqrt{\frac{1}{C} \sum_{i} (\hat{p}_i - p_i)^2 }  \\
& \text{s.t.} & & (|p_i - p_{i-1}|, |b_i - b_{i-1}|) \in \text{SLOT}_j \\
& & & C = \text{card}(\text{SLOT}_j)
\end{aligned}
\end{equation}
\myupdate{With revised \text{RMSE}, we can easily explore how the noise jumps affects the predicted price accuracy.}

\textbf{Results.} \myfigref{fig:chapter5:predicting-price} shows the different RMSE for three methods mentioned above. 
Comparing \myfigref{fig:chapter5:predicting-price-baseline} and \myfigref{fig:chapter5:predicting-price-hnn}, we see that the RMSE are close to each other when $\Delta p < 0.059$. \myupdate{But} HNN performs better than the native methods when $\Delta p > 0.059$ (\myupdate{The RMSE of HNN and native approach are 0.082 and 0.097 respectively}). It indicates HNN can predict the dramatic fluctuation of the price whereas the native approach cannot.
We observe that the results generated by HNN have smaller RMSE among most slots in \myfigref{fig:chapter5:predicting-price-lstm,fig:chapter5:predicting-price-hnn}. As for $\Delta p >= 0.059$, LSTM exceeds HNN if we only consider RMSE. However, when we inspect the predicted price obtained by HNN and LSTM, we find that HNN gives us an insight for the avalanche of the price (see \myfigref{fig:chapter5:predicting-price,fig:chapter5:price-predictions,fig:chapter5:predicting-price-2}). Briefly explanation, we see there might be an avalanche from \myfigref{fig:chapter5:predicting-price-156} if external agents sell stocks to the market. \myupdate{And in the consecutive time steps, we do observe circumstance during the simulation \myfigref{fig:chapter5:predicting-price-157}.} However, LSTM is unaware to this situations. Instead, HNN captures this signal and predicts that the price would rise in the future. And \myfigref{fig:chapter5:predicting-price-158-price} proves the predictions. However, RMSE cannot reflect this situation well and that is why the RMSE of HNN is higher than that of LSTM when $\Delta b_n$ is larger ($\Delta b_n \ge  26.2$ in the \myfigref{fig:chapter5:predicting-price}).

\begin{figure}[htb!]
    \centering
    \subfloat[Native]{
    \includegraphics[height=4cm,width=\textwidth/3]{thesis/img/predictions/baseline-rmse.png}
        \label{fig:chapter5:predicting-price-baseline}

    }
     \subfloat[LSTM]{
    \includegraphics[height=4cm,width=\textwidth/3]{thesis/img/predictions/lstm-rmse.png}
        \label{fig:chapter5:predicting-price-lstm}
    
    }
     \subfloat[HNN]{
    \includegraphics[height=4cm,width=\textwidth/3]{thesis/img/predictions/hnn-rmse.png}
            \label{fig:chapter5:predicting-price-hnn}

    }
    \caption[The RMSE for different methods.]{The RMSE for different methods. The x-axis is $\Delta p$, the y-axis is $\Delta b$, and the z-axis is RMSE.}
    \label{fig:chapter5:predicting-price}
\end{figure}


\begin{figure}[ht!]
    \centering
    \subfloat[]{
    \includegraphics[height=4cm,width=\textwidth/2]{thesis/img/price-predictions/156.pdf}
        \label{fig:chapter5:predicting-price-156}

    }
     \subfloat[]{
    \includegraphics[height=4cm,width=\textwidth/2]{thesis/img/price-predictions/157.pdf}
        \label{fig:chapter5:predicting-price-157}
    
    }

    \caption[The dynamics of aggregated agents extracted from the simulation.]{\myupdate{Two consecutive} dynamics of aggregated agents extracted from the simulation. The price changes from $p_{i-1}$ to $p_i$ in \myfigref{fig:chapter5:predicting-price-156} and jumps from $p_i$ to $p_{i+1}$ in \myfigref{fig:chapter5:predicting-price-157}. \myfigref{fig:chapter5:predicting-price-156} and \myfigref{fig:chapter5:predicting-price-157} are corresponding to \myfigref{fig:chapter5:predicting-price-157-price} and \myfigref{fig:chapter5:predicting-price-158-price} respectively. \myupdate{The blue curve (dynamics generated by \myupdate{actual scenario}) is the dynamics of aggregated agents that took place in simulation. The orange one (dynamics generated by \myupdate{potential scenario}) is that we want to inspect what the dynamics would be if external agents take the opposite action at the same time steps, i.e., if the blue curve is resulted in by external agents buy the stocks from market, the orange curve is resulted in by the external agents sell the stock to market correspondingly.}}
    \label{fig:chapter5:price-predictions}
\end{figure}

\begin{figure}[ht!]
    \centering

     \subfloat[]{
    \includegraphics[height=4cm,width=\textwidth]{thesis/img/price-predictions/157-price.pdf}
        \label{fig:chapter5:predicting-price-157-price}
    }
    \hfill
    
    \subfloat[]{
    \hbox{\hspace{0.7em}}\includegraphics[height=4cm,width=\textwidth]{thesis/img/price-predictions/158-price.pdf}
            \label{fig:chapter5:predicting-price-158-price}

    }
    \caption[The consecutive predicted prices given by HNN and LSTM networks.]{The consecutive predicted prices given by HNN and LSTM networks. \myfigref{fig:chapter5:predicting-price-156-price} and \myfigref{fig:chapter5:predicting-price-157-price} are corresponding to \myfigref{fig:chapter5:predicting-price-156} and \myfigref{fig:chapter5:predicting-price-157} respectively. \myfigref{fig:chapter5:predicting-price-158-price} is the predicted price after \myfigref{fig:chapter5:predicting-price-157-price}. The blue horizontal line is the ground-truth price, the red horizontal one is the price predicted by LSTM, the light green curve (not the horizontal green line) is the price predicted by HNN 100 times, and the green horizontal line is the average of price predicted by HNN.}
    \label{fig:chapter5:predicting-price-2}
\end{figure}
% \subsection{Real-world data set}

we use sp data set and use the sp data.

the data set is shown 
preprocessing


% \mytodo{add real world data set}

% We scale the the results by $\frac{x-\mu}{\sigma}$

% \mytodo{HERE is the sigma is 50, we will try sigma is 10 throughout this paper}
