\chapter{Evaluation\label{cha:chapter5}}

% \section{Attribute to compare}\label{}
% \mytodo{complexity of synthesis data set (nb plays/nb units/random weights/noise)} 
% \mytodo{number of sequence length (points)} 
% \mytodo{number of parameters to use in the network, our network is smaller and can  achieve the same performance as (lstm/rnn/gru)} 
% \mytodo{training loss curve} 
% \mytodo{time consumption during train/prediction} 
% \mytodo{(mean square error/cross entropy) on prediction data set}
In this chapter, we evaluate different methods on hysteretical data sets and compare the performance among them.

\section{Data sets}
We generate data sets to identify whether LSTM network and HNN could capture the micro loops of a hysteretical process. 
\newline
\textbf{Data set $D$}. The set $D$ (see \myfigref{fig:chapter5:dima-seq}) consists 1000 points. We repeat sequence $0, 3, 5, 0, 1, 5$ to generate the input $x(n)$ for $D$. We also insert $0, -100$ at the very beginning of the input $x(n)$ in order to erase the memory in the hysteretical process. The first 600 points is used as training set and the rest 400 points is test set. We scale the input of test set by 1/10, 1/7, 1/6, 1/5, 1/4, 1/3, 1/1, 1/0.5 every 50 points in order to identify inner loops.  
\newline
\textbf{Data set $P$} As for the set $P$ (see \myfigref{fig:chapter5:pavel-seq}), it also contains 1000 points. The inputs $x(n)$ is sampled from a periodical function $5cos(0.1n)$ with random noise $noise$. Repeatedly, $noise$ is drawn from different normal distributions, which standard deviations are $0.1, 0.5, 1, 2, 3, 4, 5$. 
\newline
\textbf{Data set $S$}. We generate set $S$ (see \myfigref{fig:chapter5:s-seq}) with different number of points. The inputs $x(n)$ of $S$ is sampled from a periodical signal function $\cos(0.1n) + 0.3\sin(1.3n) + 1.2\sin(0.6n)$. 
\newline
\textbf{Data set $M$} \mytodo{the outputs is a random walk}

% \begin{figure}[h!]
%     \centering
%     \subfloat[]{
%     \includegraphics[height=7cm,width=\textwidth]{thesis/img/debug-input-output-dima.pdf}
%     }
%     \caption{Data set D. The first and second plots are inputs vs. time step $n$ and outputs vs. time step $n$, respectively. Time step $n$ is between 10 and 1000. The last one is outputs vs. inputs, only showing $y(n) \in [-0.5, 45]$}
%     \label{fig:chapter5:dima-seq}
% \end{figure}

% \begin{figure}[h!]
%     \centering
%     \includegraphics[height=7cm,width=\textwidth]{thesis/img/debug-input-output-pavel.pdf}
%     \caption{Data set P. The first and second plots are inputs vs. time step $n$ and outputs vs. time step $n$, respectively. Time step $n$ is from 0 to 1000. The last one is outputs vs. inputs, only showing $y(n) \in [-0.5, 45]$}
%     \label{fig:chapter5:pavel-seq}
% \end{figure}

% \begin{figure}[h!]
%     \centering
%     \subfloat[]{
%         \includegraphics[height=5cm,width=\textwidth]{thesis/img/debug-input-output-sin-2.pdf}
%     }
%     \hfill
%     \subfloat[]{
%         \includegraphics[height=5cm,width=10cm]{thesis/img/debug-input-output-sin.pdf}
%     }
%     \caption{Data set S. The first and second plots are inputs vs. time step $n$ and outputs vs. time step $n$, respectively. Time step $n$ is from 0 to 1000.}
%     \label{fig:chapter5:s-seq}
% \end{figure}

\begin{figure}
    \centering
    \subfloat[]{
        \includegraphics[height=3cm,width=\textwidth]{debug-input-output-dima-2.pdf}
    }
    \hfill
    \subfloat[]{
        \includegraphics[height=3cm,width=\textwidth]{debug-input-output-pavel-2.pdf}
    }
    \hfill
    \subfloat[]{
        \includegraphics[height=3cm,width=\textwidth]{debug-input-output-sin-2.pdf}
    }
    \hfill
    \subfloat[]{
        \includegraphics[width=\textwidth/3]{debug-input-output-dima.pdf}
    }
    \subfloat[]{
        \includegraphics[width=\textwidth/3]{debug-input-output-pavel.pdf}
    }
    \subfloat[]{
        \includegraphics[width=\textwidth/3]{debug-input-output-sin.pdf}
    }
    \caption{Data sets.}
    \label{fig:chapter5:data-sets}
\end{figure}

\section{Architectures}
\textbf{LSTM network.} We use one LSTM layer with tanh nonlinearities and one dense layer. We use the default settings of hyper parameters in tensorflow \citep{abadi2016tensorflow} except the hidden units $units$ in LSTM layer. We grid search a sequence of hidden units, $1, 8, 16, 32, 64, 128, 256$, to find out the best results achieved by LSTM networks. As for the loss functions, we minimize mean squared error (MSE) and minimize negative maximal likelihood estimator (MLE). The total number of parameters used in this architecture is given by
\begin{equation*}
    \text{\#parameters} = 4 * ({\text{\#\_\_units\_\_}}^2 + 2*\text{\#\_\_units\_\_}) + \text{\#\_\_units\_\_} + 1
\end{equation*}
% where $\#units$ is the number of hidden units.

\textbf{HNN.}  We use architecture shown in \mychapterref{cha:chapter1}, see \myfigref{fig:chapter1:nn-arch}. We also exploit hyper parameters $plays$. $units$ in HNN. The number of parameters $\#parameters$ used in HNN is given by
\begin{equation*}
    \text{\#parameters} = 3 * \text{\#\_\_plays\_\_} * \text{\#\_\_units\_\_} + 1
\end{equation*}
% where $\#play$ is the number of plays and $\#units$ is the number of units.

\textbf{Hyper parameters}
In this evaluation, we focus on optimizing the following hyper parameters in LSTM networks and HNN.
\begin{table}[h!]
\begin{center}
    \begin{tabular}{||c|c||}
    \hline
    Name                  & Explanation \\
    \hline 
    \#nb\_plays           & the number of $plays$ in ground-truth data sets \\
    \hline
    \#units               & the number of $units$ in ground-truth data sets \\
    \hline
     \#\_\_nb\_plays\_\_  & the number of $plays$ used in training phase \\
     \hline
     \#\_\_units\_\_      & the number of $units$ used in training phase \\
    \hline
    \end{tabular}
    \caption{Hyper parameters used for evaluation}
\end{center}
\end{table}

\section{Measure}
The overall \textit{root mean squared error} (RMSE) is used to measure the quality of fit if loss function is MSE. It's defined as following,
% \begin{equation}
% \text{RMSE} = \sqrt{\frac{\sum_{i=0}^{N-1}(\hat{y}_i - y_i)^2}{N}}
% \end{equation}
% Additionally, we defined RMSE(n) as
\begin{equation}
    \text{RMSE(n)} = \sqrt{\frac{\sum_{i=n}^{N-1}(\hat{y}_i - y_i)^2}{N-n}}, \quad n=0,1,\ldots,N-1
\end{equation}
where $n$ means the first $(n-1)$th data points are ignored in RMSE results.

Especially, we set $\text{RMSE}=\text{RMSE(0)}$ when $n=0$.

\mytodo{For MLE, we use alternative criteria to measure the result of mu and sigma.}
\section{Micro experiments}
Before we move the evaluation the performance of HNN, we want to show that the initial states of HNN don't matter if we have enough large data sets. Also, we will show the fact that $G^{-1}$ can be approximate by $F$ within tolerant error.

\subsection{Initial states}
We use data set $P$ and set different initial states in predicting phase.
\mytableref{tbl:chapter5:initial-states} shows the influence of different initial states. We see incorrect initial states have bad impacts on RMSE. But these impacts cannot be ignored after 150 time steps. From \myfigref{fig:chapter5:initial-states}, we see that the predictive curves with different initial states are overlapping exactly after 150 time steps. 

In the following analysis, we always set the correct initial states for predicting phase.

\begin{table}[htb!]
\centering
\begin{adjustbox}{angle=0}
\begin{tabular}{||c|c|c|c|c||}
\hline 
    Data sets & Initial state & RMSE & RMSE(50) & RMSE(150) \\
\hline \hline
\multirow{4}{4em}{$P$} & 1 & 4.97 $\pm$ 0.04 &  2.52 $\pm$ 0.03 & 0.15 $\pm$ 0.03 \\
                           & 100 & 4.97 $\pm$ 0.04 & 2.52 $\pm$ 0.03 & 0.15 $\pm$ 0.03 \\
                           & -1 & 0.69 $\pm$ 0.04 & 0.51 $\pm$ 0.03 & 0.15 $\pm$ 0.03 \\
                           & -100 & 2.90 $\pm$ 0.03 & 1.20 $\pm$ 0.01 & 0.15 $\pm$ 0.03 \\
                           & \textbf{correct} & \textbf{0.15 $\pm$ 0.04} & \textbf{0.15 $\pm$ 0.03} & \textbf{0.15 $\pm$ 0.03} \\
\hline
\hline
\end{tabular}
\end{adjustbox}
\caption{Influence of initial state for predictions}
\label{tbl:chapter5:initial-states}
\end{table}


\begin{figure}[h!]
    \centering
    \includegraphics[height=5cm,width=\textwidth]{thesis/img/debug-initial-state-pavel-seq.pdf}
    \caption{Different initial states for set $P$}
    \label{fig:chapter5:initial-states}
\end{figure}

\subsection{Inverse of HNN}

% We call HNN as $G$ and the inverse function of $G$ is $F$.
% In this section, we show that the inverse $G$ can be approximated by another network $F$ (trained as \ref{xxx}). 
% is an experiment to see whether HNN is inverse or not.

% Before we move the evaluation the performance of HNN, we want to show that the initial states of HNN don't matter if we have enough large data sets. Also, we will show the fact that $G^{-1}$ can be approximate by $F$ within tolerant error.

% \textbf{Data sets.} In order to show the HNN is inverse, we generate two data sets by $G$ networks and switch the $outputs$ and $inputs$ of them, we call $\bar{P}$ and $\bar{S}$. Finally, we train HNN network $F$ using method shown in \mysectionref{sec:chapter2:training-pi-network}.
% $\bar{P}$ respectively.
\textbf{Setup.} One uses the data sets $S$ and $P$ generated from $G$ network and switch the $inputs$ and $outputs$ of them. We call these new data sets $\bar{S}$ and $\bar{P}$ respectively. One can train the $F$ network with $\bar{S}$ and $\bar{P}$ by training approach provided in \mysectionref{sec:chapter3:training-pi-network} and compare the $outputs$ of $F$ network with $inputs$ in $G$ network.
\newline
\textbf{Results.} \mytableref{tbl:chapter5:inverse-of-HNN} shows \mytodo{analysis here}.  
\myfigref{fig:chapter5:inverse-hnn-confidence-band} shows that all prediction results tightly bound 

\begin{table}[htb!]
\centering
\begin{adjustbox}{angle=0}
\begin{tabular}{||c|c|c|c||}
\hline 
Data sets & Length & hyper parameters & RMSE \\
\hline \hline
\multirow{3}{4em}{$\bar{S}$ set} & \multirow{1}{4em}{100} & (50, 50) & 0 $\pm$ 0 \\ 
                          \cline{2-4}
                          
                          & \multirow{1}{4em}{1000}  & (50, 50) & 0 $\pm$ 0 \\ 
                          \cline{2-4}
                          
                          & \multirow{1}{4em}{5000}  & (50, 50) & 0 $\pm$ 0 \\ 
                          \cline{2-4}
                          
% \hline                          
% \multirow{3}{4em}{$\bar{P}$ set} & \multirow{1}{4em}{100}  & (50, 50) & 0 $\pm$ 0 \\ 

%                           \cline{2-4}
                          
%                           & \multirow{1}{4em}{1000} & (50, 50) &  0.35 $\pm$ 0.06 \\ 
                      
%                           \cline{2-4}
                          
%                           & \multirow{1}{4em}{5000}  & (50, 50) & 0 $\pm$ 0 \\ 
%                           \cline{2-4}                          
\hline
\hline
\end{tabular}
\end{adjustbox}
\caption{RMSE for the data sets $\bar{S}$}
\label{tbl:chapter5:inverse-of-HNN}
\end{table}

\begin{figure}[htb!]
    \centering
    \subfloat[]{
    \includegraphics[width=\textwidth/3]{thesis/img/inverse-hnn-pavel-seq-__units__-25-__nb_plays__-25.pdf}
    }
    \subfloat[]{
    \includegraphics[width=\textwidth/3]{thesis/img/inverse-hnn-pavel-seq-__units__-25-__nb_plays__-25.pdf}
    }
    \subfloat[]{
    \includegraphics[width=\textwidth/3]{thesis/img/inverse-hnn-pavel-seq-__units__-25-__nb_plays__-25.pdf}
    }
    \caption{The results of confidence band. The curve in red ground-truth $inputs$ in $G$ network and the region in light blue is the maximal and minimal prediction of $F$ network at each time step.}
    \label{fig:chapter5:inverse-hnn-confidence-band}
\end{figure}
\section{Hysteretic loop}

We \myupdate{train} each network 20 times \myupdate{on train sets with different random seeds} and calculate the average and the standard deviation of metric \myupdate{on test sets} to \myupdate{see} if a network produces stable results.
\mytableref{tbl:chapter5:dima-pavel-seq-results} shows the metric of the different methods. We exploit hyper parameters and select the best results achieved by those methods.
We mark a result in bold if \myupdate{RMSE} is 
%\todo{I think if the RMSE if 5 times less than the other one, then it's significantly better.} 
significantly \myupdate{smaller} than other methods. We see that the HNN network achieves the best RMSE on all the data sets, which indicates it fits the hysteretic process best. Additionally, the amount of parameters HNN used is only about $1/9$ than that LSTM networks used. 

\myfigref{fig:chapter5:dima-seq-prediction-outputs-vs-time-steps} and \myfigref{fig:chapter5:pavel-seq-prediction-outputs-vs-time-steps} show the predictions on set $D$ and $P$ at each time step respectively. The larger blue areas indicate the higher uncertainty of the predictions, which means the performance of the network is worse. 

In \myfigref{fig:chapter5:dima-lstm-results}, we see the predictions of LSTM networks perform extremely poor considering the \myupdate{time steps from 601 to 900} corresponding to the micro-loops. The main reason is that LSTM networks do not take hysteretic properties into consideration. On the contrary, the blue areas restricted around the ground-truth curve tightly reveal that the HNN can reconstruct micro-loops from the training data set without observing them before. We can see that both LSTM networks and HNN predict well for time steps ranges from 900 to 950 since these data are presented in the training set. For the last 50 time steps, it corresponds to the macro-loops (see \myfigref{fig:chapter1:hysteresis-loop}) in the hysteretic process. Similarly, we see that HNN outperforms LSTM networks.

For the performance of set $P$, we see that the blue areas in \myfigref{fig:chapter5:pavel-lstm-results} are significantly larger than those in \myfigref{fig:chapter5:pavel-hnn-results}, especially for the first 100 time steps in test set. We see the noise is spread in the training set evenly (repeating with standard deviation $0.1, 0.5, 1, 2, 3, 4, 5$ every seven steps.) in set $P$. However, the first 100 data points are mainly dominated by $5 \cos(0.1n)$ due to the small standard deviation ($\sigma=0.1$), which is profoundly distinct from the training set. LSTM networks cannot deal with such a distribution in the test set well and predict results with high uncertainty. Whereas, HNN treats the training set as a hysteretic process and inspects the intrinsic properties of this dynamics successfully. That is why the blue areas of HNN are tightly adhered to with the ground-truth curve. 

\myfigref{fig:chapter5:dima-hnn-lstm-dynamics} and \myfigref{fig:chapter5:pavel-hnn-lstm-dynamics} show the \myupdate{trajectory} of set $D$ and $P$ individually. We apply cubic interpolation algorithm (see \myappendixsectionref{appendix:interpolation}) on $inputs$ of test set and generate new interpolated $inputs$. After that, we feed the interpolated $inputs$ into the trained networks to generate the interpolated predictive results. 

For data set $D$, we see the red curve (generated by LSTM networks) is oscillated left and right horizontally. But, the green curve  (generated by HNN) can follow the dynamics of the blue one in \myfigref{fig:chapter5:dima-hnn-lstm-dynamics-1,fig:chapter5:dima-hnn-lstm-dynamics-2,fig:chapter5:dima-hnn-lstm-dynamics-3,fig:chapter5:dima-hnn-lstm-dynamics-4,fig:chapter5:dima-hnn-lstm-dynamics-5}. We emphasize that the micro-loops in \myfigref{fig:chapter5:dima-hnn-lstm-dynamics-1,fig:chapter5:dima-hnn-lstm-dynamics-2,fig:chapter5:dima-hnn-lstm-dynamics-3,fig:chapter5:dima-hnn-lstm-dynamics-4,fig:chapter5:dima-hnn-lstm-dynamics-5} aren't exposed to the model during training. That's why the dynamic curve of LSTM networks is a horizontal line. Conversely, HNN reconstructs this kind of hysteretic process properly from the training set even though there still remains some bias. Particularly, \myfigref{fig:chapter5:dima-hnn-lstm-dynamics-9} is associated with the dynamics of the last 50 points in the test set, \myupdate{which corresponds to macro-loops}, we see that HNN is also able to reconstruct the macro-loops. 

As for data set $P$, we see the green curve (generated by HNN) and blue one (ground-truth curve) are overlapping precisely. At the same time, the red curve, which is generated by LSTM networks, deviates from the trace of the blue one. That confirms HNN does capture in micro-loops in a hysteretic process, but LSTM networks fail. 
Interestingly, \myfigref{fig:chapter5:pavel-hnn-lstm-dynamics-1} and \myfigref{fig:chapter5:pavel-hnn-lstm-dynamics-2} show that the red curve oscillates around the middle of the micro loop. It implies the fact that LSTM networks try to predict the average of a hysteretic process.

% \mytodo{to update, both predict based on the training set.}
% Overall, the HNN inspects the hysteretic data set and reconstructs this process correctly. Whereas, LSTM networks predict the results based on what it has seen in the training set (see \myfigref{fig:chapter5:dima-hnn-lstm-dynamics-8,fig:chapter5:dima-lstm-results}) and fail to rebuild the micro-loops and macro-loops.
\myupdate{Overall, the HNN inpsects the hysteretic data set and reconstructs macro-loops and micro-loops correctly. Whereas, LSTM networks fails to reconstruct this process due to lack of the assumptions on hysteretic data sets. Moreover, HNN fits the data sets better with fewer parameters compared with LSTM/SimpleRNN/GRU.}

\begin{table}[htb!]
\centering
\begin{adjustbox}{angle=0}
\begin{tabular}{||c|c|cc||}
\hline 
Data sets & networks & \#parameters & RMSE \\
\hline \hline
\multirow{2}{4em}{$D$ set} 
                             & \multirow{1}{6em}{SimpleRNN} & 16769 & 12.62 $\pm$ 1.50  \\             
\cline{2-4}

                             & \multirow{1}{6em}{GRU} & 50049 & 12.13 $\pm$ 2.64 \\ 
\cline{2-4}
                             & \multirow{1}{6em}{LSTM} & 66560 & 14.39 $\pm$ 4.07 \\
\cline{2-4}
                             & \multirow{1}{6em}{HNN} & 7501 & \textbf{2.79 $\pm$ 0.09} \\ 
\hline \hline
\multirow{2}{4em}{$P$ set} 

% 
                            & \multirow{1}{6em}{SimpleRNN} & 16769 & 5.75 $\pm$ 0.66 \\
\cline{2-4}

                             & \multirow{1}{6em}{GRU} & 50049 & 4.70 $\pm$ 0.45 \\ 
\cline{2-4}
                            & \multirow{1}{6em}{LSTM}   & 66560 & 4.99 $\pm$ 0.33 \\ 

\cline{2-4}
                         & \multirow{1}{6em}{HNN} 
                         & 7501 & \textbf{0.09 $\pm$ 0.02} \\ 
\hline
\end{tabular}
\end{adjustbox}
\caption[The RMSE for the test sets $D$ and $P$]{The RMSE for the test sets $D$ and $P$. The total number of parameters used to generate ground-truth is 7501 (50 $nb\_plays$ and 50 $units$).}
\label{tbl:chapter5:dima-pavel-seq-results}
\end{table}

\begin{figure}[h!]
    \centering
    \subfloat[LSTM]{
        \includegraphics[width=\textwidth/2]{thesis/img/debug-lstm-dima-__units__-128.pdf}
        \label{fig:chapter5:dima-lstm-results}
    }
    \subfloat[HNN]{
        \includegraphics[width=\textwidth/2]{thesis/img/debug-hnn-dima-__units__-25-__nb_plays__-25.pdf}
        \label{fig:chapter5:dima-hnn-results}
    }
    \caption[Predictions for test set $D$.]{Predictions for test set $D$. The curve in red is the ground-truth test set and the light blue areas correspond to the uncertainty of the predictive results. \myfigref{fig:chapter5:dima-lstm-results} is the predictions of LSTM networks and \myfigref{fig:chapter5:dima-hnn-results} is the predictions of HNN.}
    \label{fig:chapter5:dima-seq-prediction-outputs-vs-time-steps}
\end{figure}

\begin{figure}[h!]
    \centering
    \subfloat[LSTM]{
        \includegraphics[width=\textwidth]{thesis/img/debug-lstm-pavel-__units__-128.pdf}
        \label{fig:chapter5:pavel-lstm-results}
    }
    \hfill
    \subfloat[HNN]{
        \includegraphics[width=\textwidth]{thesis/img/debug-hnn-pavel-__units__-25-__nb_plays__-25.pdf}
        \label{fig:chapter5:pavel-hnn-results}
    }
    \caption[Predictions for test set $P$.]{Predictions for test set $P$. The curve in red is the ground-truth test set and the light blue areas correspond to the uncertainty of the predictive results. \myfigref{fig:chapter5:pavel-lstm-results} is the predictions of LSTM networks and \myfigref{fig:chapter5:pavel-hnn-results} is the predictions of HNN.}
    \label{fig:chapter5:pavel-seq-prediction-outputs-vs-time-steps}
\end{figure}


\begin{figure}[h]
    \subfloat[]{
    \includegraphics[width=\textwidth/3]{debug-dima-hnn-lstm-inspection/1.pdf}
    \label{fig:chapter5:dima-hnn-lstm-dynamics-1}

    }
    \subfloat[]{
    \includegraphics[width=\textwidth/3]{debug-dima-hnn-lstm-inspection/2.pdf}
    \label{fig:chapter5:dima-hnn-lstm-dynamics-2}
    }
    \subfloat[]{
    \includegraphics[width=\textwidth/3]{debug-dima-hnn-lstm-inspection/3.pdf}
    \label{fig:chapter5:dima-hnn-lstm-dynamics-3}
    
    }
    \hfill
    \subfloat[]{
    \includegraphics[width=\textwidth/3]{debug-dima-hnn-lstm-inspection/4.pdf}
    \label{fig:chapter5:dima-hnn-lstm-dynamics-4}

    }
    \subfloat[]{
    \includegraphics[width=\textwidth/3]{debug-dima-hnn-lstm-inspection/5.pdf}
    \label{fig:chapter5:dima-hnn-lstm-dynamics-5}

    }
    \subfloat[]{
    \includegraphics[width=\textwidth/3]{debug-dima-hnn-lstm-inspection/6.pdf}
    \label{fig:chapter5:dima-hnn-lstm-dynamics-6}
    
    }
    \hfill
    \subfloat[]{
    \includegraphics[width=\textwidth/3]{debug-dima-hnn-lstm-inspection/7.pdf}
    \label{fig:chapter5:dima-hnn-lstm-dynamics-7}
    
    }
    \subfloat[]{
    \includegraphics[width=\textwidth/3]{debug-dima-hnn-lstm-inspection/8.pdf}
    \label{fig:chapter5:dima-hnn-lstm-dynamics-8}

    }
    \subfloat[]{
    \includegraphics[width=\textwidth/3]{debug-dima-hnn-lstm-inspection/9.pdf}
    \label{fig:chapter5:dima-hnn-lstm-dynamics-9}
    
    }
    \caption[The animation of hysteretic loops for set $D$. ]{The animation of hysteretic loops for set $D$. \myupdate{Different plots correspond to different time steps. The last time step of the (blue/green/red) curves in each plot is monotonically increasing. The scatter points in cyan consist of ground-truth and predicted test sets.} The curve in blue is the animation of ground-truth data sets and the one in red is the animation generated by LSTM networks and the one in green is the animation generated by HNN.}
    \label{fig:chapter5:dima-hnn-lstm-dynamics}
\end{figure}


%% lstm & hnn motion
\begin{figure}[h]
    \subfloat[]{
    \includegraphics[width=\textwidth/3]{debug-pavel-hnn-lstm-inspection/1.pdf}
    \label{fig:chapter5:pavel-hnn-lstm-dynamics-1}

    }
    \subfloat[]{
    \includegraphics[width=\textwidth/3]{debug-pavel-hnn-lstm-inspection/2.pdf}
    \label{fig:chapter5:pavel-hnn-lstm-dynamics-2}
    }
    \subfloat[]{
    \includegraphics[width=\textwidth/3]{debug-pavel-hnn-lstm-inspection/3.pdf}
    \label{fig:chapter5:pavel-hnn-lstm-dynamics-3}
    
    }
    \hfill
    \subfloat[]{
    \includegraphics[width=\textwidth/3]{debug-pavel-hnn-lstm-inspection/4.pdf}
    \label{fig:chapter5:pavel-hnn-lstm-dynamics-4}

    }
    \subfloat[]{
    \includegraphics[width=\textwidth/3]{debug-pavel-hnn-lstm-inspection/5.pdf}
    \label{fig:chapter5:pavel-hnn-lstm-dynamics-5}

    }
    \subfloat[]{
    \includegraphics[width=\textwidth/3]{debug-pavel-hnn-lstm-inspection/6.pdf}
    \label{fig:chapter5:pavel-hnn-lstm-dynamics-6}
    
    }
    \hfill
    \subfloat[]{
    \includegraphics[width=\textwidth/3]{debug-pavel-hnn-lstm-inspection/7.pdf}
    \label{fig:chapter5:pavel-hnn-lstm-dynamics-7}
    
    }
    \subfloat[]{
    \includegraphics[width=\textwidth/3]{debug-pavel-hnn-lstm-inspection/8.pdf}
    \label{fig:chapter5:pavel-hnn-lstm-dynamics-8}

    }
    \subfloat[]{
    \includegraphics[width=\textwidth/3]{debug-pavel-hnn-lstm-inspection/9.pdf}
    \label{fig:chapter5:pavel-hnn-lstm-dynamics-9}
    
    }
    \caption[The animation of hysteretic loops for set $P$.]{The animation of hysteretic loops for set $P$. \myupdate{Different plots correspond to different time steps. The last time step of the (blue/green/red) curves in each plot is monotonically increasing. The scatter points in cyan consist of ground-truth and predicted test sets.} The curve in blue is the animation of ground-truth data sets and the one in red is the animation generated by LSTM networks and the one in green is the animation generated by HNN.}
    \label{fig:chapter5:pavel-hnn-lstm-dynamics}
\end{figure}

\FloatBarrier
\section{MLE}

\textbf{Data set.} outputs is a random walk and input is generated from inverse network $F$

% \mytodo{same weights but different nb plays}
% \mytodo{different weights and different plays  no noise}
% \mytodo{different weights  and different plays noise} 
% \section{hnn-lstm-all-sigma-0}

\begin{table}[h]
\begin{adjustbox}{angle=0}
\begin{tabular}{||c|c|c|c||}
\hline 
\#play (\#unit) & network & \#parameters & RMSE 
\\ \hline
%%%%%%%%%%%%%%%%%%%%%%%%%%%%%%%%%%%%%%%%%%%%%%
1& lstm-unit-1 & 12 & 0.293196103
\\ \hline
1 & lstm-unit-8 & 320 & 0.3420867
\\ \hline
1 & lstm-unit-16 & 1,152 & 0.288819476
\\ \hline
1 & lstm-unit-32 & 4,352 & 0.291840419
\\ \hline
1 & lstm-unit-64 & 16,896 & 0.453519368
\\ \hline
1 & lstm-unit-128 & 66,560 &  0.261012763
\\ \hline
1 & lstm-unit-256 & 264,192 & 0.387604944
\\ \hline
%%%%%%%%%%%%%%%%%%%%%%%%%%%%%%%%%%%%%%%%%%%%%%
50 & lstm-unit-1 & 12 & 7.24246318
\\ \hline
50 & lstm-unit-8 & 320 & 11.47440343
\\ \hline
50 & lstm-unit-16 & 1,152 & 6.699965269
\\ \hline
50 & lstm-unit-32 & 4,352 & 5.361344497
\\ \hline
50 & lstm-unit-64 & 16,896 & 6.257610363
\\ \hline
50 & lstm-unit-128 & 66,560 & 16.18529624
\\ \hline
50 & lstm-unit-256 & 264,192 & 14.22657381
\\ \hline
%%%%%%%%%%%%%%%%%%%%%%%%%%%%%%%%%%%%%%%%%%%%%%
100 & lstm-unit-1 & 12 & 23.56603552
\\ \hline
100 & lstm-unit-8 & 320 & 7.714270118
\\ \hline
100 & lstm-unit-16 & 1,152 & 9.293488583
\\ \hline
100 & lstm-unit-32 & 4,352 & 8.239636524
\\ \hline
100 & lstm-unit-64 & 16,896 & 42.83023104
\\ \hline
100 & lstm-unit-128 & 66,560 &  40.50239738
\\ \hline
100 & lstm-unit-256 & 264,192 & 31.05811984
\\ \hline

%%%%%%%%%%%%%%%%%%%%%%%%%%%%%%%%%%%%%%%%%%%%%%
% 500 & lstm-unit-1 & -1 & 14.27020212
% \\ \hline
% 500 & lstm-unit-8 & -1 & 15.04781055
% \\ \hline
% 500 & lstm-unit-16 & -1 & 18.72035688
% \\ \hline
% 500 & lstm-unit-32 & -1 & 13.1013827
% \\ \hline
% 500 & lstm-unit-64 & -1 & 42.8302338
% \\ \hline
% 500 & lstm-unit-128 & -1 &  40.50240013
% \\ \hline
% 500 & lstm-unit-256 & -1 & 40.50240013
% \\ \hline

\end{tabular}
\end{adjustbox}
\caption{lstm same weights no noise}
\end{table}

% currently, it is given by the following formula
% \begin{equation}
%         \#parameter = \#play * (3\#unit+1)
% \end{equation}

\begin{table}[h]
\begin{adjustbox}{angle=0}
\begin{tabular}{||c|c|c|c||}
\hline 
\#play (\#unit) & network & \#parameters & RMSE 
\\ \hline
%%%%%%%%%%%%%%%%%%%%%%%%%%%%%%%%%%%%%%%%%%%%%%
1& hnn-play-1-unit-1 & 4 & 0.181846233
\\ \hline
1 & hnn-play-2-unit-2 & 13 & 0.18374151
\\ \hline
1 & hnn-play-8-unit-8 & 193 & 0.168357833
\\ \hline
%%%%%%%%%%%%%%%%%%%%%%%%%%%%%%%%%%%%%%%%%%%%%%

50 & hnn-play-50-unit-50 & 7,501 & 6.352061288
\\ \hline
50 & hnn-play-100-unit-100 & 30,001 & 6.18795001

\\ \hline
50 & hnn-play-200-unit-200 & 120,001 &  6.163790714

\\ \hline
%%%%%%%%%%%%%%%%%%%%%%%%%%%%%%%%%%%%%%%%%%%%%%

100 &  hnn-play-50-unit-50 & 7,501 & 12.71635473
\\ \hline
100 &  hnn-play-100-unit-100 & 30,001 & 12.42587169

\\ \hline
100 &  hnn-play-200-unit-200 & 120,001 & 11.97417496
\\ \hline
\end{tabular}
\end{adjustbox}
\caption{hnn same weights no noise using elu as activation}
\end{table}
% \section{different-weights-no-noise, sigma-0}
\begin{table}
\begin{adjustbox}{angle=0}
\begin{tabular}{||c|c|c|c||}
\hline 
\#play (\#unit) & network & \#parameters & RMSE 
\\ \hline
%%%%%%%%%%%%%%%%%%%%%%%%%%%%%%%%%%%%%%%%%%%%%%
50 & lstm-unit-1 & 12 & 2.210005733
\\ \hline
50 & lstm-unit-8 & 320 & 2.96759788

\\ \hline
50 & lstm-unit-16 & 1,152 & 2.138047736

\\ \hline
50 & lstm-unit-32 & 4,352 & 2.385471522

\\ \hline
50 & lstm-unit-64 & 16,896 & 2.60343689

\\ \hline
50 & lstm-unit-128 & 66,560 &  2.176426255

\\ \hline
50 & lstm-unit-256 & 264,192 & 2.176366421

\\ \hline
%%%%%%%%%%%%%%%%%%%%%%%%%%%%%%%%%%%%%%%%%%%%%%
100 & lstm-unit-1 & 12 & 4.792876647

\\ \hline
100 & lstm-unit-8 & 320 & 3.100921422

\\ \hline
100 & lstm-unit-16 & 1,152 & 1.459150006

\\ \hline
100 & lstm-unit-32 & 4,352 & 2.49986648

\\ \hline
100 & lstm-unit-64 & 16,896 & 2.514736698

\\ \hline
100 & lstm-unit-128 & 66,560 & 2.105919619

\\ \hline
100 & lstm-unit-256 & 264,192 & 2.105532431
\\ \hline
\end{tabular}
\end{adjustbox}
\caption{lstm different weights no noise}
\end{table}


\begin{table}
\begin{adjustbox}{angle=0}
\begin{tabular}{||c|c|c|c||}
\hline 
\#play (\#unit) & network & \#parameters & RMSE 
\\ \hline
% %%%%%%%%%%%%%%%%%%%%%%%%%%%%%%%%%%%%%%%%%%%%%%
% 1& hnn-play-1-unit-1 & 4 & 0.181846233
% \\ \hline
% 1 & hnn-play-2-unit-2 & 13 & 0.18374151
% \\ \hline
% 1 & hnn-play-8-unit-8 & 193 & 0.168357833
% \\ \hline
%%%%%%%%%%%%%%%%%%%%%%%%%%%%%%%%%%%%%%%%%%%%%%
50 & hnn-play-10-unit-10 & 301 & 0.520737239
\\ \hline
50 & hnn-play-25-unit-10 & 751 & 0.481243223

\\ \hline
50 & hnn-play-25-unit-25 & 1,876 & 0.657608419

\\ \hline
50 & hnn-play-50-unit-50 & 7,501 &  0.678940872

\\ \hline
50 & hnn-play-100-unit-50 & 15,001 &  0.669122054


\\ \hline
%%%%%%%%%%%%%%%%%%%%%%%%%%%%%%%%%%%%%%%%%%%%%%

100 &  hnn-play-25-unit-25 & 1,876 & 0.650535543

\\ \hline
100 &  hnn-play-50-unit-25 & 3,751 & 0.618666728

\\ \hline
100 &  hnn-play-50-unit-50 & 7,501 & 0.642434578

\\ \hline
100 &  hnn-play-100-unit-50 & 15,001 & 0.65746967

\\ \hline
100 &  hnn-play-100-unit-100 & 30,001 & 0.639300454

\\ \hline
100 &  hnn-play-200-unit-25 & 15,001 & 0.343120708

\\ \hline
100 &  hnn-play-200-unit-100 & 60,001 & 0.466267496

\\ \hline
\end{tabular}
\end{adjustbox}
\caption{hnn different weights no noise using elu as activation}
\end{table}

% \section{different-weights-with-noise, sigma-8}
% \begin{table}
% \begin{adjustbox}{angle=0}
% \begin{tabular}{||c|c|c|c||}
% \hline 
% \#play (\#unit) & network & \#parameters & RMSE 
% \\ \hline
% %%%%%%%%%%%%%%%%%%%%%%%%%%%%%%%%%%%%%%%%%%%%%%
% 50 & lstm-unit-1 & 12 & 3.444246389
% \\ \hline
% 50 & lstm-unit-8 & 320 & 1.463447021
% \\ \hline
% 50 & lstm-unit-16 & 1,152 & 1.328417165
% \\ \hline
% 50 & lstm-unit-32 & 4,352 & 1.303146381
% \\ \hline
% 50 & lstm-unit-64 & 16,896 & 1.201215367
% \\ \hline
% 50 & lstm-unit-128 & 66,560 &  1.398202932
% \\ \hline
% 50 & lstm-unit-256 & 264,192 & 1.257714804
% \\ \hline
% %%%%%%%%%%%%%%%%%%%%%%%%%%%%%%%%%%%%%%%%%%%%%%
% 100 & lstm-unit-1 & 12 & 6.104827801
% \\ \hline
% 100 & lstm-unit-8 & 320 & 2.496286256

% \\ \hline
% 100 & lstm-unit-16 & 1,152 & 2.613680848

% \\ \hline
% 100 & lstm-unit-32 & 4,352 & 2.75013601

% \\ \hline
% 100 & lstm-unit-64 & 16,896 & 2.73756359

% \\ \hline
% 100 & lstm-unit-128 & 66,560 & 2.574973189

% \\ \hline
% 100 & lstm-unit-256 & 264,192 & 2.56076419

% \end{tabular}
% \end{adjustbox}
% \caption{lstm different weights  noise}
% \end{table}


% \begin{table}
% \begin{adjustbox}{angle=0}
% \begin{tabular}{||c|c|c|c||}
% \hline 
% \#play (\#unit) & network & \#parameters & RMSE 
% \\ \hline

% 50 & hnn-play-10-unit-10 & 301 & 0.600649707
% \\ \hline
% 50 & hnn-play-25-unit-10 & 751 & 0.590401014
% \\ \hline
% 50 & hnn-play-25-unit-25 & 1,876 & 0.640757911
% \\ \hline
% 50 & hnn-play-50-unit-50 & 7,501 &  0.570615017
% \\ \hline
% 50 & hnn-play-100-unit-50 & 15,001 &  0.582596486

% \\ \hline
% %%%%%%%%%%%%%%%%%%%%%%%%%%%%%%%%%%%%%%%%%%%%%%
% 100 &  hnn-play-25-unit-25 & 1,876 & 0.801752993
% \\ \hline
% 100 &  hnn-play-50-unit-25 & 3,751 & 1.42465427
% \\ \hline
% 100 &  hnn-play-50-unit-50 & 7,501 & 0.959867101
% \\ \hline
% 100 &  hnn-play-100-unit-50 & 15,001 & 1.985093566
% \\ \hline
% 100 &  hnn-play-100-unit-100 & 30,001 & 0.436862487
% \\ \hline
% 100 &  hnn-play-200-unit-25 & 15,001 & 1.135538106
% \\ \hline
% 100 &  hnn-play-200-unit-100 & 60,001 & 0.844494969
% \\ \hline
% \end{tabular}
% \end{adjustbox}
% \caption{hnn different weights noise using elu as activation}
% \end{table}



\begin{table}[h]
\centering
\begin{adjustbox}{angle=0}
\begin{tabular}{||c|c|ccc||}
\hline 
\#plays & networks & hyper parameters & \#parameters & RMSE ($\mu \pm \sigma$)\\
\hline \hline
\multirow{13}{4em}{50} & \multirow{7}{4em}{LSTM}   & 1    & 12         & 3.444246389 \\ 
                         &                         & 8    & 320        & 1.463447021 \\ 
                         &                         & 16   & 1,152      & 1.328417165 \\ 
                         &                         & 32   & 4,352      & 1.303146381 \\ 
                         &                         & 64   & 16,896     & 1.201215367 \\ 
                         &                         & 128  & 66,560     & 1.398202932 \\ 
                         &                         & 256  & 264,192    & 1.257714804 \\ 
\cline{2-5}
                         & \multirow{6}{4em}{HNN}  & (10, 10)    & 301      & 0.600649707 \\
                         &                         & (25, 10)    & 751      & 0.590401014 \\ 
                         &                         & (25, 25)    & 1,876    & 0.640757911 \\ 
                         &                         & (50, 25)    & 7,501    & 0.570615017 \\ 
                         &                         & (100, 50)   & 15,001   & 0.582596486 \\ 
                         
\hline \hline
\multirow{13}{4em}{100} & \multirow{7}{4em}{LSTM}  & 1    & 12          & 6.104827801 \\ 
                         &                         & 8    & 320         & 2.496286256 \\ 
                         &                         & 16   & 1152        & 2.613680848 \\ 
                         &                         & 32   & 4352        & 2.75013601  \\ 
                         &                         & 64   & 16896       & 2.73756359  \\ 
                         &                         & 128  & 66560       & 2.574973189 \\ 
                         &                         & 256  & 264192      & 2.56076419  \\ 
\cline{2-5}
                         & \multirow{6}{4em}{HNN}  & (25, 25)     & 1,876    & 0.801752993 \\
                         &                         & (50, 50)     & 7,501    & 0.959867101 \\ 
                         &                         & (100, 50)    & 15,001   & 1.985093566 \\ 
                         &                         & (100, 100)   & 30,001   & 0.436862487 \\        
                         &                         & (200, 25)    & 15,001   & 1.135538106 \\        
                         &                         & (200, 100)   & 60,001   & 0.844494969 \\        

\hline
\end{tabular}
\end{adjustbox}
\caption{RMSE for the data sets D and P}
\end{table}
\section{HNN for market model}\label{}
% \mytodo{hnn for markov chain dataset}
% \mytodo{compare network complexity, mle loss} 
% \mytodo{compare the markov chain obtain by lstm and hnn}
% \mytodo{show animation that lstm performs worse due to cannot capture minor loops.}
\mytodo{show that hnn can reconstruct agents behavior}
\mytodo{show that hnn can give a distribution of price, not average of price} 
\mytodo{show that hnn can explain the avalanche of market} 
\mytodo{add real world dataset}

We scale the the results by $\frac{x-\mu}{\sigma}$

\mytodo{HERE is the sigma is 50, we will try sigma is 20 throughout this paper}
% \section{Synthetic data sets}\label{sec:chapter3:synthetic-datasets}
% Following the practical mentioned in previous section, we obtain the following artificial data sets. See \myfigref{fig:chapter3:agents-distribution,fig:chapter3:market-participanted-agents,fig:chapter3:market-ground-truth-dataset}
% % \mytodo{add distribution of real agent D, virtual agent N}

% \begin{figure}[htb!]
%     % \left
%     %\subfloat[]{
%         %\raisebox{0px} {
%             % \adjustimage{width=\textwidth/2}{
%             \includegraphics[height=3cm,width=\textwidth/2]{market-real-agents-n-distribution}\label{fig:chapter3:market-real-agents-n-distribution}
%             \includegraphics[height=3cm,width=\textwidth/2]{market-virtual-agents-d-distribution}\label{fig:chapter3:market-virtual-agents-d-distribution}
             
%             % \adjustimage{width=\textwidth/2}{market-virtual-agents-d-distribution}\label{fig:chapter3:market-virutal-agents-d-distribution}

%         % }
%     % }

%     % \subfloat[]{
%     %         \includegraphics[width=\textwidth/2]{market-virtual-agents-d-distribution}
%     %                                 \label{fig:chapter3:market-virutal-agents-d-distribution}

%     % }
%     % \hfill
%     %     \subfloat[]{
%              % \adjustimage{width=\textwidth/2,left}{market-real-agents-alpha-distribution}
%              % \label{fig:chapter3:market-real-agents-alpha-distribution}
%             \includegraphics[height=3cm,width=\textwidth/2]{market-real-agents-alpha-distribution}\label{fig:chapter3:market-real-agents-alpha-distribution}

%     % }
%     \caption{\myfigref{fig:chapter3:market-real-agents-n-distribution} shows the distribution of real agents N. \myfigref{fig:chapter3:market-real-agents-alpha-distribution} shows the distribution of $\alpha$ for each agent. \myfigref{fig:chapter3:market-virutal-agents-d-distribution} shows the distribution of virtual agents D}
%     \label{fig:chapter3:agents-distribution}
% \end{figure}

% \begin{figure}[t!]
%     \centering
%     \subfloat[]{
%     \includegraphics[height=7cm,width=\textwidth]{market-participanted-agents}
%     }
%     \caption{The detailed behaviours of agents taken part in the market during simulation}
%     \label{fig:chapter3:market-participanted-agents}
% \end{figure}

% \begin{figure}[t!]
%     \centering
%     \subfloat[]{
%         \includegraphics[height=7cm,width=\textwidth]{market-ground-truth-dataset}
%     }
%     \caption{The synthetic data sets generated from the model. The top one is the random walk, which is underline in the real market, of the simulation. The middle one is the fluctuation of price based on the model. And the bottom one the change of the total number of stocks in the market, which is following to random walk.}
%     \label{fig:chapter3:market-ground-truth-dataset}
% \end{figure}