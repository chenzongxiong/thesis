\chapter{Summary and future works\label{cha:chapter6}}

The objective of this thesis is to approximate a broad class of hysteretic systems by neural networks. In \mychapterref{cha:chapter2}, we proposed new network architecture, so-called hysteretic neural network (HNN), which is a realization of a linear combination of multiple nonlinear \textit{plays}. Then, we formulated an agent-based financial market model as an illustrative application and proposed a method to learn this financial model by HNN in \mychapterref{cha:chapter3} and \mychapterref{cha:chapter4}.

We evaluate both HNN and RNN to learn from the synthetic data sets and compare their performance. The experiments highlight that HNN is a computationally efficient and accurate alternative for solving the dynamics on networks of PI operators of arbitrary complexity and with arbitrary inputs. Analysis of the financial model learned by HNN and LSTM reveals that HNN achieves lower RMSE and reconstructs the agents' behaviors in the proposed market models.

Unfortunately, we did not exploit multiple tuning techniques for the neural networks' hyperparameters. Further studies include trying different hyperparameters to check the performance of the proposed methods. Moreover, we can evaluate a real-world data set, such as S\&P price sequence, and compare the performance of HNN with state-of-the-art networks.
